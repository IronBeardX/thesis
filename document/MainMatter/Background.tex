\chapter{Estado del Arte}\label{chapter:state-of-the-art}

La revisión de la literatura científica es un componente fundamental en cualquier investigación académica, ya que permite situar el trabajo en el contexto del conocimiento existente, identificar avances recientes y reconocer las limitaciones o brechas que aún persisten. En este capítulo, se realiza un análisis detallado del estado del arte relacionado con la predicción epidémica, haciendo énfasis en los métodos tradicionales y modernos, como el uso de \textit{GNN's}  y \textit{Transfer Learning}.

El objetivo principal de este capítulo es proporcionar una base conceptual y técnica que sustente el desarrollo del modelo propuesto en esta tesis. Para ello, se abordan tres áreas principales: Los enfoques tradicionales y basados en aprendizaje automático para la predicción de brotes, las aplicaciones específicas de GNNs en este ámbito, y el rol del \textit{Transfer Learning} como técnica para superar limitaciones de datos en escenarios específicos. Adicionalmente, se incluye un análisis crítico de estudios recientes relevantes para este trabajo y se destacan las oportunidades de investigación que guían el desarrollo de esta tesis.

Este capítulo está estructurado de la siguiente manera: en la Sección 1.1, se revisan los métodos tradicionales de predicción epidémica, sus aplicaciones y limitaciones. La Sección 1.2 introduce las técnicas basadas en aprendizaje automático, con un énfasis particular en las redes neuronales profundas. En la Sección 1.3, se exploran las arquitecturas y aplicaciones de GNNs en modelación epidemiológica, mientras que en la Sección 1.4 se analiza la relevancia del \textit{Transfer Learning} y su implementación en GNNs.

Este análisis no solo busca sintetizar el conocimiento disponible, sino también resaltar las áreas donde se requiere mayor investigación, particularmente en contextos con recursos limitados y características socioeconómicas particulares, como es el caso de Cuba.

\section{Métodos Tradicionales de Predicción Epidémica}\label{section:traditional-methods}

La predicción de epidemias ha sido una prioridad en la investigación epidemiológica, especialmente durante los brotes de enfermedades infecciosas, como el dengue, la fiebre amarilla, y más recientemente COVID-19. Tradicionalmente, los modelos de predicción epidémica se han basado en métodos matemáticos y estadísticos que intentan capturar la dinámica de propagación de enfermedades a través de diferentes enfoques\parencite{Kermack1927ACT, Chowell2016EarlyGrowth, Burke2024OriginsSEIR}. En este sentido, se han propuesto varios tipos de modelos para la predicción y el control de brotes epidémicos, destacándose los modelos de compartimentos\parencite{Kermack1927ACT, Chowell2016EarlyGrowth, Mata2021MathematicalEpidemics}, los modelos basados en agentes y los enfoques estadísticos\parencite{Rodriguez2022DataCentric, Nowzari2016ComplexNetworks}.

En este apartado, se proporciona una visión general de los métodos tradicionales más utilizados para la predicción epidémica. A continuación, se presenta una clasificación gráfica de estos métodos, seguida de un análisis detallado de cada uno de ellos, sus ventajas y limitaciones, con el fin de establecer un contexto para los métodos basados en \textit{GNNs}  y \textit{Transfer Learning}, que serán discutidos más adelante\parencite{Nguyen2023NZSTGNN, Zheng2024HeatGNN}.

\begin{figure}[H]
\centering
%TODO: \includegraphics[width=0.8\textwidth]{traditional\_methods\_classification.png}
\caption{Clasificación de los métodos tradicionales de predicción epidémica\parencite{Mata2021MathematicalEpidemics, Rodriguez2022DataCentric}.}
\end{figure}

\subsection{Modelos de Compartimentos: El Modelo SIR y sus Variantes}\label{section:sir-model}

El modelo \textit{SIR} (Susceptibles, Infectados y Recuperados) es uno de los enfoques más antiguos y utilizados en la epidemiología matemática para modelar la propagación de enfermedades infecciosas. Introducido por Kermack y McKendrick en 1927\parencite{Kermack1927ACT}, este modelo divide a la población en tres compartimentos básicos: Susceptibles, Infectados y Recuperados. Cada compartimento representa una etapa en la que un individuo se encuentra en relación con la enfermedad, y las personas pasan de un estado a otro conforme interactúan con individuos infectados. Este modelo es particularmente útil para describir enfermedades con un período de incubación, en las cuales un individuo infectado pasa por una fase \textit{expuesta} antes de volverse infeccioso\parencite{Mata2021MathematicalEpidemics}. 

La estructura básica del modelo SIR se describe mediante un sistema de ecuaciones diferenciales ordinarias (EDOs), donde la población total se distribuye entre los cuatro compartimentos mencionados. La dinámica de estas transiciones se expresa de la siguiente forma:

\[
\frac{dS}{dt} = - \frac{\beta}{N}IS
\]
\[
\frac{dI}{dt} = \frac{\beta}{N}IS - \gamma I
\]
\[
\frac{dR}{dt} = \gamma I
\]

En este sistema, \(S\) representa a la población susceptible a la enfermedad, \(I\) a los individuos infectados e infecciosos, y \(R\) a los individuos que se han recuperado o han muerto. Los parámetros \(\beta\) y \(\gamma\) representan las tasas de transmisión, la tasa de transición del compartimento expuesto al infectado, y la tasa de recuperación de los infectados, respectivamente.

El modelo SIR es particularmente adecuado para enfermedades con un período de incubación conocido, como \textit{COVID-19}\parencite{Burke2024OriginsSEIR}, en el que se sabe que el período de incubación varía entre 2 y 14 días. Este modelo ha sido ampliamente utilizado en la predicción de la propagación de diversas enfermedades infecciosas debido a su simplicidad y a la facilidad con que se puede ajustar a diferentes contextos epidemiológicos.

A pesar de su simplicidad, el modelo SIR se basa en varias suposiciones que limitan su aplicabilidad en contextos más complejos\parencite{Chowell2016EarlyGrowth}. En primer lugar, asume que toda la población es homogénea, lo que significa que todos los individuos tienen la misma probabilidad de ser infectados, sin considerar las diferencias en edad, comportamiento social o condiciones socioeconómicas. Además, el modelo SIR supone que los contactos entre individuos son aleatorios y homogéneos, lo cual no refleja la estructura de redes sociales complejas en la que las personas interactúan de manera no uniforme\parencite{Moein2021SIRInefficiency}. Tampoco toma en cuenta las características espaciales y geográficas de la población, que son fundamentales para modelar la propagación de enfermedades en contextos como el de Cuba, donde existen diferencias significativas en densidad poblacional y acceso a servicios de salud\parencite{Nowzari2016ComplexNetworks}.

A lo largo del tiempo, han surgido diversas variantes del modelo SIR con el objetivo de superar algunas de estas limitaciones. Una de las principales modificaciones es el modelo SIRS (Susceptible, Expuesto, Infectado, Recuperado, Susceptible), que introduce la posibilidad de que los individuos recuperados puedan volver a ser susceptibles después de un cierto período de tiempo, lo que es característico de enfermedades en las que la inmunidad es temporal, como ocurre con la gripe estacional\parencite{Mata2021MathematicalEpidemics}. Esta variante permite modelar con mayor precisión enfermedades cuya inmunidad no dura toda la vida, lo cual es relevante en el caso de enfermedades como la \textit{influenza}.

Otro modelo que ha ganado relevancia en las últimas décadas es el SIR con vacunación. En esta variante, se incorpora un compartimento adicional para individuos vacunados, lo que permite evaluar el impacto de las campañas de vacunación en la propagación de enfermedades. El modelo SIR con vacunación se utiliza para predecir cómo las tasas de inmunización pueden modificar la dinámica de la transmisión\parencite{Datilo2019EpidemicForecasting, Mata2021MathematicalEpidemics}, un enfoque esencial para entender la efectividad de las políticas de vacunación masiva, como las implementadas en respuesta a la pandemia de \textit{COVID-19}. En este modelo, la tasa de vacunación se integra en las ecuaciones, permitiendo calcular la proporción de la población que se vuelve inmune debido a la vacunación.

Por otro lado, el modelo SIR espacialmente explícito introduce un enfoque que tiene en cuenta las diferencias geográficas y de movilidad en la población. En lugar de asumir que la población es homogénea, este modelo permite representar a los individuos en diferentes localizaciones geográficas y modelar los movimientos entre estas localizaciones, lo que es esencial en contextos donde la propagación de la enfermedad depende fuertemente de la movilidad de las personas, como ocurre en los países insulares o en regiones con alta interacción entre zonas urbanas y rurales\parencite{Nowzari2016ComplexNetworks}. Esta variante se ha vuelto crucial para la modelización de enfermedades en contextos globalizados y urbanizados, donde las interacciones entre diferentes áreas geográficas tienen un impacto significativo en la propagación del patógeno.

Aunque las variantes del modelo SIR ofrecen mayor flexibilidad y permiten una representación más precisa de la propagación de las enfermedades en escenarios complejos, aún existen importantes limitaciones. La principal de ellas es que estos modelos siguen basándose en suposiciones de homogeneidad en ciertos aspectos, como la tasa de contacto o la movilidad, lo cual puede no ser realista en poblaciones diversas\parencite{Moein2021SIRInefficiency}. Además, la falta de datos detallados sobre la estructura social y las interacciones específicas entre subgrupos dentro de la población limita la capacidad de estos modelos para predecir con precisión la propagación de enfermedades en contextos de alta heterogeneidad\parencite{Shinde2020ForecastingCOVID}.

Otro desafío importante es la incapacidad de estos modelos para integrar en tiempo real intervenciones como el distanciamiento social, el cierre de fronteras o las campañas de vacunación. Aunque se han propuesto modificaciones para incorporar estas intervenciones, la efectividad de las mismas depende de una respuesta dinámica que pueda adaptarse rápidamente a los cambios en la propagación de la enfermedad\parencite{Shinde2020ForecastingCOVID}.

\subsection{Modelos Basados en Agentes}\label{section:agent-based-models}

Los \textit{Modelos Basados en Agentes} (ABMs) han emergido como una herramienta poderosa en la predicción y modelado de epidemias, especialmente cuando se desea capturar la heterogeneidad espacial y temporal de las interacciones entre individuos. A diferencia de los modelos compartimentales como el SIR, que tratan a la población como homogénea\parencite{Moein2021SIRInefficiency}, los ABMs modelan a los individuos como entidades autónomas con características específicas, que interactúan entre sí dentro de un entorno determinado\parencite{Chowell2016EarlyGrowth}. Estas interacciones pueden ser influenciadas por factores como el comportamiento social, la ubicación geográfica, las políticas públicas, y la movilidad, permitiendo que los ABMs simulen de manera más realista la propagación de enfermedades en contextos complejos\parencite{Mata2021MathematicalEpidemics, Nowzari2016ComplexNetworks}.

\subsubsection{Estructura Básica de un Modelo Basado en Agentes}

En un ABM, cada individuo es representado como un \"agente\" con atributos específicos, como la edad, el estado de salud, la ubicación y el comportamiento. Los agentes interactúan entre sí de acuerdo con reglas predefinidas, las cuales pueden ser estocásticas o determinísticas\parencite{Mata2021MathematicalEpidemics}. Estas interacciones afectan la propagación de la enfermedad dentro de la población. Los agentes pueden moverse de un lugar a otro, contagiarse, infectar a otros individuos, recuperarse, o incluso morir, dependiendo de las reglas que rigen el modelo\parencite{Chowell2016EarlyGrowth}.

La principal ventaja de los ABMs es que permiten capturar la heterogeneidad individual, lo que significa que los agentes pueden tener diferentes probabilidades de infección dependiendo de sus características personales o de su entorno. Además, los ABMs permiten representar de forma explícita la estructura social y geográfica de la población, lo cual es crucial para entender cómo las enfermedades se propagan en entornos con redes de contactos complejas\parencite{Nowzari2016ComplexNetworks}.

En la práctica, los ABMs se implementan mediante la simulación de miles o millones de agentes que interactúan según las reglas definidas. Estas simulaciones pueden llevarse a cabo en espacios discretos o continuos, y los modelos pueden incluir elementos como redes sociales, vecindarios, y comportamientos de movilidad, lo que permite una representación detallada de la propagación epidémica\parencite{Datilo2019EpidemicForecasting, Shinde2020ForecastingCOVID}.

\subsubsection{Aplicaciones de los Modelos Basados en Agentes en la Predicción Epidémica}

Los ABMs han sido utilizados en una variedad de contextos para modelar la propagación de enfermedades. Uno de los ejemplos más conocidos es el uso de ABMs para simular la propagación de la \textit{gripe} y el \textit{COVID-19}, donde los modelos incorporan factores como la movilidad de la población, las políticas de salud pública y las interacciones sociales. En un estudio realizado por \textcite{Mata2021MathematicalEpidemics}, se utilizaron ABMs para modelar la propagación de la influenza en una población ficticia, considerando factores como la densidad de población, las tasas de contacto entre diferentes grupos de edad y las intervenciones de salud pública como la vacunación.

Los ABMs han demostrado ser especialmente útiles en contextos donde los modelos compartimentales tradicionales, como el SIR, no pueden capturar adecuadamente la complejidad de la propagación epidémica. Por ejemplo, en \textit{modelos de propagación espacial}, los ABMs pueden simular cómo la enfermedad se extiende de una región a otra a medida que las personas se desplazan entre diferentes áreas geográficas. Estos modelos son particularmente importantes en países con grandes diferencias regionales en cuanto a infraestructura de salud y densidad poblacional, como en el caso de Cuba, donde las interacciones sociales y los movimientos entre áreas urbanas y rurales pueden tener un impacto significativo en la dinámica de propagación de enfermedades\parencite{Chowell2016EarlyGrowth, Moein2021SIRInefficiency}.

En el caso del \textit{COVID-19}, los ABMs han sido utilizados para evaluar el impacto de las políticas de distanciamiento social y otras intervenciones de salud pública. Los estudios han mostrado cómo las restricciones en la movilidad de los individuos pueden reducir significativamente la propagación del virus, pero también han destacado los retos en la implementación de estas medidas, especialmente en entornos con alta densidad de población o con movimientos transitorios (por ejemplo, turistas)\parencite{Nowzari2016ComplexNetworks}. Un estudio destacado realizado por \textcite{Datilo2019EpidemicForecasting} utilizó ABMs para simular la propagación del COVID-19 en diferentes escenarios de distanciamiento social, ayudando a comprender cómo las intervenciones pueden mitigar la propagación de la enfermedad en distintos contextos sociales.

\subsubsection{Ventajas de los Modelos Basados en Agentes}

Una de las principales ventajas de los ABMs es su capacidad para representar la \textit{heterogeneidad} de las poblaciones. A diferencia de los modelos de compartimentos, donde todos los individuos se tratan como iguales, los ABMs permiten que cada agente tenga características únicas que afectan su probabilidad de contagio y de interacción. Por ejemplo, los agentes pueden ser clasificados según su edad, sexo, nivel de salud, comportamientos sociales y actividades diarias, lo que permite simular de manera más realista cómo las enfermedades se propagan a través de diferentes grupos dentro de la población\parencite{Chowell2016EarlyGrowth, Rodriguez2022DataCentric, Mata2021MathematicalEpidemics}.

Otra ventaja importante de los ABMs es su \textit{flexibilidad} para incorporar una amplia gama de factores que afectan la propagación de la enfermedad. Los ABMs pueden incluir reglas para representar comportamientos de movilidad, redes sociales, contactos entre individuos, políticas públicas de control (como cuarentenas o vacunación) y variabilidad en la tasa de transmisión. Esta capacidad para incorporar múltiples factores hace que los ABMs sean particularmente útiles para modelar escenarios complejos donde otros métodos, como los modelos SIR, no son suficientemente detallados\parencite{Mata2021MathematicalEpidemics, Nowzari2016ComplexNetworks, Chowell2016EarlyGrowth}.

Además, los ABMs permiten realizar \textit{simulaciones de políticas públicas} de manera más precisa. Por ejemplo, se pueden simular diferentes escenarios de intervención, como la implementación de cuarentenas, la introducción de medidas de distanciamiento social o la distribución de vacunas, y evaluar el impacto de estas políticas en la propagación de la enfermedad. Esta capacidad para probar diferentes estrategias antes de implementarlas en la realidad es crucial para una planificación de respuesta más efectiva durante brotes epidémicos\parencite{Rodriguez2022DataCentric, Shinde2020ForecastingCOVID}.

\subsubsection{Limitaciones de los Modelos Basados en Agentes}

A pesar de sus ventajas, los ABMs presentan varias limitaciones que deben ser consideradas al aplicarlos en el contexto de la predicción epidémica. Una de las principales limitaciones es su \textit{alto costo computacional}. La simulación de miles o millones de agentes que interactúan entre sí puede ser muy exigente desde el punto de vista computacional, especialmente cuando se incluyen detalles complejos como la movilidad y las redes sociales. Esto puede hacer que los ABMs sean menos viables en situaciones donde se requiere una simulación rápida o en entornos con recursos limitados\parencite{Nowzari2016ComplexNetworks}.

Además, los ABMs dependen de la disponibilidad de \textit{datos detallados} sobre la población, las interacciones sociales y las políticas de salud pública. En contextos como el de Cuba, donde los datos sobre movilidad poblacional y redes de interacción social pueden ser limitados o imprecisos, los resultados de los ABMs pueden ser menos fiables. Esto resalta la necesidad de contar con datos de alta calidad y actualizados para que los modelos sean precisos y útiles en la toma de decisiones\parencite{Rodriguez2022DataCentric}.

Otra limitación importante de los ABMs es que, aunque pueden capturar la heterogeneidad de la población, \textit{la calibración} de los modelos puede ser un desafío. La falta de datos precisos sobre la distribución de ciertos atributos de la población (como las tasas de contacto, el comportamiento social, o las características de movilidad) puede llevar a resultados erróneos o poco representativos. Esto requiere que los investigadores hagan suposiciones sobre estos factores, lo que introduce incertidumbre en las predicciones\parencite{Mata2021MathematicalEpidemics}.

\subsubsection{Conclusión}

Los modelos basados en agentes ofrecen un enfoque potente y flexible para modelar la propagación de enfermedades, especialmente en escenarios donde la heterogeneidad espacial y temporal juega un papel importante. Su capacidad para representar interacciones complejas entre individuos y su flexibilidad para incorporar múltiples factores hace que los ABMs sean una herramienta valiosa en la predicción epidémica. Sin embargo, sus limitaciones en términos de costos computacionales, disponibilidad de datos y calibración del modelo subrayan la necesidad de enfoques más eficientes y precisos, como las redes neuronales y el aprendizaje transferido, que pueden manejar estos desafíos y mejorar la precisión de las predicciones\parencite{Shinde2020ForecastingCOVID}.

\subsection{Enfoques Estadísticos}\label{section:statistical-approaches}

Los enfoques estadísticos han sido una herramienta fundamental en la predicción epidémica, especialmente cuando se dispone de datos históricos limitados o de baja calidad. Estos métodos se basan en el análisis de datos pasados para modelar la propagación de una enfermedad y realizar predicciones sobre su comportamiento futuro\parencite{Chowell2016EarlyGrowth, Mata2021MathematicalEpidemics}. Aunque estos enfoques no intentan modelar la dinámica exacta de la transmisión de la enfermedad, ofrecen una forma eficiente de realizar predicciones rápidas en escenarios donde los recursos para la modelización detallada son escasos.

Uno de los enfoques más comunes en la epidemiología estadística es el uso de \textit{modelos de regresión}. Estos modelos tratan de establecer una relación matemática entre los casos reportados de la enfermedad y una o más variables predictoras, como las condiciones climáticas, las políticas de intervención, o las características demográficas de la población\parencite{Rodriguez2022DataCentric}. Los modelos de regresión pueden ser tanto lineales como no lineales, dependiendo de la naturaleza de la relación entre las variables. Por ejemplo, en el caso del análisis de la propagación del COVID-19, los modelos de regresión logística o de Poisson han sido utilizados para predecir la tasa de infección en función de factores como la densidad poblacional, la movilidad, y las intervenciones de salud pública\parencite{Shinde2020ForecastingCOVID}.

Los \textit{modelos de series temporales} son otro enfoque estadístico ampliamente utilizado para predecir la evolución de enfermedades a lo largo del tiempo. Estos modelos analizan los patrones de datos históricos, buscando tendencias y estacionalidades en la propagación de la enfermedad. Los modelos más comunes en este contexto son los \textit{modelos ARIMA} (Autoregressive Integrated Moving Average) y sus variantes, que intentan capturar las relaciones de dependencia temporal entre los valores de los datos y las perturbaciones aleatorias\parencite{LimZohren2020TimeSeries, Chowell2016EarlyGrowth}. Los modelos ARIMA son particularmente útiles cuando se dispone de datos consistentes y de alta calidad sobre la evolución de la epidemia, como los reportes diarios de nuevos casos de infección. Sin embargo, su capacidad predictiva está limitada cuando los datos son escasos o no lineales, como ocurre en los primeros días de un brote epidémico\parencite{Chowell2016EarlyGrowth, Moein2021SIRInefficiency}.

Una extensión de los modelos de series temporales es el \textit{modelo de suavizamiento exponencial}, que también intenta modelar las series de tiempo a través de un proceso de suavizado. Este tipo de modelos es particularmente útil cuando se desea realizar predicciones de corto plazo con un énfasis en las observaciones más recientes, ya que otorgan mayor peso a los datos más cercanos al momento actual. Esta propiedad permite a los modelos de suavizamiento exponencial adaptarse rápidamente a cambios repentinos en la dinámica de la epidemia, como un incremento repentino en el número de casos\parencite{Shinde2020ForecastingCOVID, LimZohren2020TimeSeries}.

Aunque estos enfoques estadísticos han demostrado ser útiles en la predicción de la propagación de enfermedades, también presentan varias limitaciones importantes que deben tenerse en cuenta. En primer lugar, los enfoques estadísticos dependen en gran medida de la calidad y cantidad de los datos históricos disponibles. En contextos donde los datos sobre la propagación de la enfermedad son limitados o de baja calidad, los modelos estadísticos pueden no ser suficientemente precisos\parencite{Rodriguez2022DataCentric}. Además, los modelos estadísticos generalmente asumen que la relación entre las variables predictoras y la propagación de la enfermedad es constante en el tiempo, lo cual no siempre es cierto en la práctica, especialmente en situaciones dinámicas como las pandemias. Las tasas de transmisión pueden cambiar a lo largo del tiempo debido a factores como la implementación de nuevas políticas, el comportamiento social de la población, o la aparición de nuevas variantes del patógeno\parencite{Moein2021SIRInefficiency}.

Otra limitación importante de los enfoques estadísticos es su incapacidad para modelar explícitamente la heterogeneidad espacial y temporal de los brotes epidémicos. A diferencia de los modelos compartimentales como el SIR o los modelos basados en agentes, los enfoques estadísticos no incorporan factores como la movilidad de la población o las interacciones sociales, que son esenciales para comprender cómo se propaga una enfermedad en contextos geográficamente diversos\parencite{Nowzari2016ComplexNetworks}. Por ejemplo, en Cuba, la distribución geográfica de la población y las diferencias en las infraestructuras de salud entre áreas urbanas y rurales podrían tener un impacto significativo en la propagación de una enfermedad, pero este tipo de información no se considera en los modelos estadísticos convencionales.

Por último, los modelos estadísticos también enfrentan desafíos en la incorporación de intervenciones dinámicas. En escenarios de epidemias en tiempo real, las políticas de salud pública, como el distanciamiento social, las cuarentenas o la vacunación, pueden alterar significativamente la trayectoria de la epidemia. Los modelos estadísticos tradicionales no están bien equipados para manejar estos cambios dinámicos de forma flexible, ya que generalmente se ajustan a datos históricos sin considerar la capacidad de respuesta en tiempo real a las intervenciones\parencite{Shinde2020ForecastingCOVID, Rodriguez2022DataCentric}.

A pesar de estas limitaciones, los enfoques estadísticos siguen siendo una herramienta valiosa para la predicción epidémica, especialmente cuando los datos son escasos o cuando se necesitan predicciones rápidas en fases tempranas de un brote. No obstante, la creciente complejidad de las epidemias modernas, sumada a la disponibilidad de nuevos tipos de datos y técnicas computacionales, hace necesario explorar enfoques más sofisticados, como el uso de \textit{Graph Neural Networks} (GNNs) y \textit{Transfer Learning}, que permiten incorporar de manera más precisa y flexible los diversos factores espaciales, temporales y sociales en la modelización de la propagación de enfermedades infecciosas\parencite{Rodriguez2022DataCentric, Panagopoulos_Nikolentzos_Vazirgiannis_2021}.

Los enfoques estadísticos han sido fundamentales para la predicción epidémica, pero sus limitaciones, particularmente en contextos dinámicos y heterogéneos, subrayan la necesidad de métodos más avanzados. Aunque son apropiados para realizar predicciones a corto plazo basadas en datos históricos, estos enfoques carecen de la capacidad para adaptarse rápidamente a nuevas situaciones o integrar variables complejas como la movilidad poblacional o las intervenciones en tiempo real. Por lo tanto, los modelos basados en redes neuronales, que pueden manejar estos factores, representan una mejora significativa para la predicción de epidemias en escenarios como el cubano\parencite{Nguyen2023NZSTGNN, Zheng2024HeatGNN}.

\section{Limitaciones de los Métodos Tradicionales de Predicción Epidémica}

A pesar de su relevancia histórica y su aplicación extendida en la modelización epidemiológica, los métodos tradicionales de predicción, como los modelos compartimentales (por ejemplo, SIR y SEIR), presentan limitaciones significativas cuando se enfrentan a los desafíos complejos de los brotes epidémicos actuales\cite{Kermack1927ACT, Burke2024OriginsSEIR}. Estas limitaciones pueden agruparse en varias categorías clave, según lo identificado en diversos estudios recientes.

En primer lugar, los modelos compartimentales clásicos asumen parámetros constantes y homogeneidad en la población, lo cual no refleja la realidad de muchos brotes epidémicos\cite{Moein2021SIRInefficiency}. Por ejemplo, el modelo SEIR no considera adecuadamente las variaciones en las tasas de contacto debido a intervenciones como cuarentenas o distanciamiento social. Esto limita su capacidad para capturar cambios dinámicos en la propagación de enfermedades como el COVID-19, donde los parámetros de transmisión pueden fluctuar considerablemente en respuesta a políticas públicas o la aparición de nuevas variantes del virus\cite{Rizzo2020AutoSEIR, Baccega2024Sybil}.

Además, la incapacidad de estos modelos para integrar datos espaciales y temporales con suficiente granularidad representa un desafío significativo\cite{Nowzari2016ComplexNetworks}. Aunque algunas variantes espaciales de los modelos compartimentales han sido propuestas, como el modelo SIR espacialmente explícito, su aplicabilidad está restringida debido a la complejidad del cálculo y a la falta de datos precisos para parametrizar las interacciones entre diferentes regiones geográficas\cite{Mata2021MathematicalEpidemics}.

Otro problema crítico es la dificultad de incorporar características heterogéneas de las poblaciones, tales como diferencias demográficas, socioeconómicas y de comportamiento. Estas características son esenciales para comprender la propagación de enfermedades en países como Cuba, donde las disparidades entre regiones urbanas y rurales afectan significativamente las dinámicas de transmisión\cite{Rodriguez2022DataCentric, Nguyen2023NZSTGNN}. Además, los modelos compartimentales suelen depender de supuestos simplificados, como tasas de transmisión constantes y contactos homogéneos, que no reflejan las complejas interacciones de redes sociales y la movilidad humana\cite{Nguyen2023NZSTGNN}.

Desde una perspectiva computacional, los modelos tradicionales también son limitados en términos de la capacidad de predicción en tiempo real y su integración con datos fluctuantes. Los estudios recientes han señalado que estos modelos tienden a ajustarse mal cuando los datos son incompletos o están sesgados, un problema común en el monitoreo de epidemias en tiempo real\cite{Shinde2020ForecastingCOVID, Chowell2016EarlyGrowth}. Además, la dependencia de datos agregados en lugar de individuales puede resultar en predicciones menos precisas y menos útiles para la toma de decisiones en salud pública\cite{Santangelo2023MachineLearning}.

Por último, los modelos tradicionales enfrentan limitaciones en su capacidad para manejar incertidumbres inherentes a los sistemas epidemiológicos, como las mutaciones del virus y la aparición de variantes nuevas. Aunque algunos modelos estocásticos han intentado abordar estas incertidumbres, su aplicación práctica sigue siendo limitada debido a la falta de datos adecuados para validar y calibrar estos enfoques\cite{Moein2021SIRInefficiency, Chowell2016EarlyGrowth}.

Aunque los métodos tradicionales de predicción epidémica han sido fundamentales en la epidemiología, sus limitaciones en términos de homogeneidad de los supuestos, integración de datos heterogéneos y dinámica en tiempo real subrayan la necesidad de enfoques más avanzados. La integración de técnicas modernas como las redes neuronales gráficas (\textit{Graph Neural Networks}) y el aprendizaje transferido (\textit{Transfer Learning}) promete abordar estas limitaciones, ofreciendo mayor flexibilidad, precisión y adaptabilidad en escenarios complejos y dinámicos.

\section{Métodos Basados en Aprendizaje Automático}\label{section:machine-learning-methods}

El uso de técnicas de aprendizaje automático (AA) en la predicción epidémica ha ganado popularidad en los últimos años debido a su capacidad para manejar grandes volúmenes de datos, adaptarse a la dinámica cambiante de las epidemias y modelar relaciones complejas que no son fácilmente capturadas por los enfoques tradicionales. A diferencia de los métodos clásicos como los modelos de compartimentos, que requieren suposiciones previas sobre la estructura de la propagación de la enfermedad, los métodos basados en AA pueden aprender directamente de los datos, lo que los convierte en herramientas poderosas para modelar la propagación de enfermedades en contextos dinámicos y con datos incompletos o ruidosos\parencite{Datilo2019EpidemicForecasting, Stergiou2022MachineLearning}.

Existen diversos enfoques de AA que han sido aplicados al pronóstico epidémico, entre los cuales destacan los modelos de \textit{redes neuronales recurrentes (LSTM)}, los \textit{árboles de decisión aleatorios (Random Forests)} y el \textit{aprendizaje por refuerzo (Reinforcement Learning)}\parencite{Rodriguez2022DataCentric}. Cada uno de estos métodos tiene sus fortalezas y limitaciones, y su aplicabilidad depende del tipo de datos disponibles y de la naturaleza del brote epidémico que se desea modelar.

\subsection{Redes Neuronales Recurrentes (LSTM)}

Las Redes Neuronales Recurrentes (RNNs) y sus variantes, como las Redes de Memoria a Largo y Corto Plazo (\textit{Long Short-Term Memory}, LSTM), han sido ampliamente utilizadas en el campo del aprendizaje automático para modelar datos secuenciales y realizar predicciones basadas en series temporales. Su capacidad para mantener información relevante a lo largo del tiempo las hace especialmente útiles en la predicción de epidemias, donde las dependencias temporales juegan un papel crucial en la dinámica de propagación de enfermedades\parencite{Shah2024COVID19FU}.

Las LSTM se destacan por su arquitectura, que incorpora una celda de memoria diseñada para resolver el problema del desvanecimiento de gradientes que afecta a las RNN tradicionales\parencite{Santangelo2023MachineLearning}. Esta celda permite a las LSTM aprender relaciones tanto a corto como a largo plazo, capturando dinámicas complejas que son esenciales en la predicción de brotes epidémicos. En el contexto de las epidemias, las LSTM pueden procesar datos históricos de incidencia de casos, movilidad poblacional y factores socioeconómicos para predecir la evolución futura de la enfermedad\parencite{Nguyen2023NZSTGNN}.

Un ejemplo destacado del uso de LSTM es su aplicación para predecir la propagación del COVID-19 en Canadá, donde este modelo demostró una alta precisión al capturar las tendencias no lineales de los datos diarios de casos confirmados\parencite{Shah2024COVID19FU}. Otro estudio comparó LSTM con métodos tradicionales como ARIMA, encontrando que las LSTM superaron significativamente a estos modelos en términos de precisión en escenarios dinámicos y no lineales\parencite{Nguyen2023NZSTGNN, Baccega2024Sybil}. 

Además, las LSTM han sido utilizadas en combinación con otras técnicas, como redes gráficas y modelos de predicción híbridos, para capturar tanto las dependencias temporales como las relaciones espaciales. Por ejemplo, modelos como \textit{MPNN+LSTM} integran capacidades de paso de mensajes en grafos con análisis de series temporales, mejorando la precisión en predicciones multirregionales de COVID-19 en Nueva Zelanda\parencite{Nguyen2023NZSTGNN}.

\subsection{Redes Neuronales Recurrentes (RNNs) y sus Variantes (GRU, Bi-LSTM)}

Las Redes Neuronales Recurrentes (\textit{Recurrent Neural Networks}, RNNs) han sido ampliamente utilizadas en la predicción de brotes epidémicos debido a su capacidad para modelar dependencias temporales en datos secuenciales. Estas redes están diseñadas específicamente para manejar datos donde la información previa tiene un impacto directo en las predicciones futuras, como es el caso de las series temporales en epidemiología. Dentro de este marco, se han desarrollado variantes avanzadas, como las redes de memoria a largo plazo (\textit{Long Short-Term Memory}, LSTM), las unidades recurrentes controladas (\textit{Gated Recurrent Units}, GRU), y las redes bidireccionales LSTM (\textit{Bidirectional LSTM}, Bi-LSTM), que abordan algunas de las limitaciones de las RNNs tradicionales\parencite{Shah2024COVID19FU, Santangelo2023MachineLearning}.

\paragraph{RNNs Básicas}
Las RNNs convencionales emplean una estructura de bucle que les permite mantener un estado oculto que se actualiza en cada paso temporal, capturando así la información previa de la secuencia. Sin embargo, presentan limitaciones cuando las dependencias temporales abarcan largos períodos, ya que el gradiente tiende a disiparse o explotar durante el entrenamiento. Este problema ha llevado al desarrollo de arquitecturas más avanzadas, como las LSTM y las GRU\parencite{Rodriguez2022DataCentric}.

\paragraph{LSTM}
Introducidas por Hochreiter y Schmidhuber en 1997, las LSTM incorporan una memoria explícita mediante el uso de puertas de entrada, salida y olvido, lo que les permite conservar información relevante durante períodos prolongados\parencite{Santangelo2023MachineLearning}. Esta característica ha hecho que las LSTM sean especialmente adecuadas para modelar la propagación de epidemias, donde las interacciones entre eventos pasados y futuros son clave. Por ejemplo, las LSTM han sido utilizadas para predecir casos confirmados de COVID-19 en Canadá, mostrando un desempeño robusto en escenarios con datos altamente no lineales y dinámicos\parencite{Nguyen2023NZSTGNN}.

\paragraph{GRU}
Las GRU, introducidas posteriormente, simplifican la arquitectura de las LSTM al combinar las puertas de entrada y olvido en una sola unidad, lo que reduce la complejidad computacional sin sacrificar significativamente el rendimiento. Este enfoque más eficiente las hace particularmente útiles para la predicción en tiempo real en contextos epidemiológicos\parencite{Santangelo2023MachineLearning, Stergiou2022MachineLearning}.

\paragraph{Bi-LSTM}
Las redes Bi-LSTM amplían el concepto de las LSTM tradicionales al procesar la información en ambas direcciones de la secuencia temporal. Esto permite capturar relaciones tanto hacia adelante como hacia atrás en el tiempo, mejorando la precisión de las predicciones en casos donde el contexto completo de la serie temporal es importante. Estudios recientes han demostrado que las Bi-LSTM superan a las LSTM tradicionales en la predicción de casos confirmados, recuperaciones y muertes durante la pandemia de COVID-19\parencite{Shah2024COVID19FU, Baccega2024Sybil}.

\paragraph{Aplicaciones en Epidemiología}
Estas variantes de RNN han demostrado ser herramientas valiosas para modelar y predecir la dinámica de enfermedades infecciosas. Por ejemplo, las LSTM han sido utilizadas para predecir el número de casos diarios en diferentes regiones, considerando factores como la movilidad poblacional y las políticas de intervención\parencite{Rodriguez2022DataCentric}. Las GRU han mostrado su efectividad en escenarios donde la capacidad computacional es limitada, mientras que las Bi-LSTM han destacado en tareas que requieren una comprensión más profunda de las dependencias temporales, como la predicción de tendencias epidémicas durante oleadas sucesivas del virus\parencite{Nguyen2023NZSTGNN, Baccega2024Sybil}.

\subsection{Support Vector Machines (SVM)}

El método de \textit{Support Vector Machines} (SVM) es un enfoque de aprendizaje supervisado ampliamente utilizado en tareas de clasificación y regresión. Su principio fundamental radica en encontrar un hiperplano que maximice el margen entre las clases de datos en un espacio de características, lo que lo hace particularmente efectivo en problemas de alta dimensionalidad o con conjuntos de datos pequeños\parencite{Santangelo2023MachineLearning}. Las SVM son conocidas por su capacidad para manejar problemas lineales y no lineales mediante el uso de funciones kernel que proyectan los datos a espacios de mayor dimensión, donde las relaciones complejas entre las variables pueden ser más fácilmente separadas\parencite{Stergiou2022MachineLearning}.

En el contexto de la predicción epidémica, las SVM se han aplicado principalmente para modelar la incidencia y las tendencias de enfermedades infecciosas como el COVID-19. Estas aplicaciones incluyen la predicción del número de casos confirmados, la identificación de áreas con alto riesgo de transmisión, y el análisis de factores que influyen en la propagación de enfermedades. Por ejemplo, estudios recientes han demostrado que las SVM son efectivas para integrar datos epidemiológicos, como la densidad poblacional y las tasas de transmisión, y generar predicciones robustas basadas en patrones históricos\parencite{Santangelo2023MachineLearning, Rodriguez2022DataCentric}.

Un aspecto destacado de las SVM en el análisis epidemiológico es su capacidad para trabajar con datos heterogéneos y no estructurados, como información demográfica, registros clínicos y datos de movilidad\parencite{Shah2024COVID19FU}. Esto permite capturar relaciones complejas entre múltiples factores que afectan la propagación de enfermedades infecciosas, ofreciendo una ventaja sobre los métodos tradicionales. Además, las SVM han sido utilizadas como componente en enfoques híbridos que combinan modelos matemáticos compartimentales, como SEIR, con algoritmos de aprendizaje automático para mejorar la precisión de las predicciones\parencite{Baccega2024Sybil, Stergiou2022MachineLearning}.

Estudios como los de \textit{Santangelo et al.} han explorado cómo las SVM pueden ser utilizadas para realizar predicciones de corto plazo en escenarios dinámicos, mostrando que estos métodos no solo son efectivos en la predicción de casos, sino también en la identificación de factores clave que impulsan la transmisión de enfermedades\parencite{Santangelo2023MachineLearning, Rodriguez2022DataCentric}. Por ejemplo, las SVM han sido utilizadas para predecir la incidencia del COVID-19 en varias regiones del mundo, demostrando su capacidad para modelar tanto datos lineales como no lineales en diferentes contextos geográficos y temporales\parencite{Shah2024COVID19FU}.

\subsection{Árboles de Decisión Aleatorios (Random Forests)}\label{section:random-forests}

Los \textit{Árboles de Decisión Aleatorios} (Random Forests, RF) son un método de aprendizaje supervisado basado en ensamblajes de árboles de decisión. Este enfoque combina múltiples árboles entrenados en diferentes subconjuntos del conjunto de datos para producir una predicción final más robusta y precisa. En el contexto de la predicción epidémica, los RF han demostrado ser una herramienta valiosa para modelar relaciones no lineales complejas y capturar interacciones entre múltiples variables predictoras\parencite{Stergiou2022MachineLearning}.

Un ejemplo destacado del uso de RF en la predicción epidémica es su aplicación en el modelado de brotes de enfermedades como el dengue y el COVID-19. En el caso del dengue, los RF han sido utilizados para predecir la probabilidad de brotes en función de variables climáticas, densidad poblacional y datos de movilidad. Este enfoque permite identificar regiones con mayor riesgo de transmisión y proporciona información clave para la planificación de intervenciones de salud pública\parencite{Nguyen2023NZSTGNN, Baccega2024Sybil}.

En el contexto del COVID-19, los RF han sido empleados para predecir tanto la incidencia de casos como la evolución a corto plazo de la enfermedad. Por ejemplo, un estudio reciente utilizó RF para modelar la propagación del COVID-19 en Europa, integrando datos demográficos, tasas de infección, y características socioeconómicas. Los resultados mostraron que este enfoque era particularmente útil para capturar patrones de propagación en áreas con datos heterogéneos y dinámicas no lineales\parencite{Santangelo2023MachineLearning, Rodriguez2022DataCentric}.

Los RF son también reconocidos por su capacidad para manejar datos con alta dimensionalidad, lo que es especialmente relevante en el análisis epidemiológico donde los conjuntos de datos pueden incluir múltiples variables, desde factores biológicos hasta datos geográficos y socioeconómicos. Esta capacidad les permite realizar predicciones confiables incluso en contextos donde los datos son ruidosos o incompletos\parencite{Saleem2022MLDLModels}.

Otra ventaja importante de los RF en la predicción epidémica es su habilidad para evaluar la importancia relativa de las variables predictoras. Esto permite a los investigadores identificar los factores más influyentes en la propagación de una enfermedad, como la movilidad interregional o las tasas de vacunación, lo que puede informar estrategias de control y mitigación más efectivas\parencite{Rizzo2020AutoSEIR, Baccega2024Sybil}.

\subsection{Autoregressive Integrated Moving Average (ARIMA)}

El modelo \textit{Autoregressive Integrated Moving Average} (ARIMA) es una técnica estadística ampliamente utilizada en la predicción de series temporales. En el contexto de la predicción epidémica, ARIMA se emplea para modelar el comportamiento de una enfermedad basándose en datos históricos, identificando patrones en la evolución de los casos reportados y realizando predicciones a corto y mediano plazo. Este modelo se fundamenta en tres componentes principales: autorregresión (AR), diferencia integrada (I) y promedio móvil (MA), los cuales permiten capturar tanto la relación entre valores pasados como las tendencias y fluctuaciones en los datos\parencite{LimZohren2020TimeSeries, Datilo2019EpidemicForecasting}.

El ARIMA ha sido ampliamente aplicado en la modelación de epidemias debido a su capacidad para trabajar con datos agregados y proporcionar predicciones rápidas. Por ejemplo, el modelo ha demostrado ser efectivo para pronosticar la evolución diaria de casos confirmados de COVID-19 en múltiples países, particularmente durante las primeras etapas de la pandemia, cuando los datos eran limitados y las decisiones rápidas eran esenciales\parencite{Chowell2016EarlyGrowth, Baccega2024Sybil}. Asimismo, investigaciones recientes han utilizado ARIMA para analizar la relación entre factores externos, como condiciones meteorológicas, y la propagación de enfermedades, mostrando resultados prometedores en la integración de datos contextuales en los pronósticos\parencite{Baccega2024Sybil}.

En términos de implementación, el ARIMA requiere una serie temporal estacionaria para generar predicciones precisas. Esto se logra a través de la diferenciación, un proceso que elimina tendencias y fluctuaciones estacionales. Adicionalmente, el ajuste de los parámetros del modelo (\(p\), \(d\), \(q\))—representando el orden de la autorregresión, el número de diferencias integradas, y el orden del promedio móvil, respectivamente—es fundamental para su desempeño. En la práctica, estos parámetros se determinan comúnmente mediante el análisis de autocorrelación y autocorrelación parcial, lo que permite identificar las características de la serie temporal y seleccionar el modelo más adecuado para los datos\parencite{Saleem2022MLDLModels, Santangelo2023MachineLearning}.

El ARIMA también ha sido combinado con otros enfoques, como el uso de datos epidemiológicos provenientes de modelos compartimentales, para mejorar su capacidad predictiva. Por ejemplo, en estudios recientes, se integraron modelos SEIR con ARIMA para estimar parámetros como el número reproductivo efectivo (\(R_t\)) y realizar predicciones más precisas durante los picos de la pandemia\parencite{Baccega2024Sybil, Rodriguez2022DataCentric}. Estas integraciones destacan la flexibilidad del ARIMA para adaptarse a contextos complejos y dinámicos.

\subsection{Modelo Prophet}\label{section:prophet-model}

El modelo \textit{Prophet} es una herramienta desarrollada por Facebook para la predicción de series temporales, diseñada específicamente para capturar estacionalidad, tendencias y eventos atípicos en los datos. Este enfoque se basa en un modelo aditivo donde la serie temporal \( y(t) \) se descompone en tres componentes principales: una tendencia \( g(t) \), una estacionalidad \( s(t) \), y una función \( h(t) \) para manejar eventos atípicos:

\[
y(t) = g(t) + s(t) + h(t) + \epsilon_t
\]

donde \( \epsilon_t \) representa el error irreducible en el modelo. La flexibilidad de Prophet radica en su capacidad para manejar de manera efectiva series temporales con datos irregulares o estacionales marcados, lo que lo hace particularmente adecuado para escenarios de predicción epidemiológica\parencite{Baccega2024Sybil}.

El modelo ha sido ampliamente utilizado en la predicción del COVID-19 debido a su facilidad para incorporar tendencias a largo plazo y eventos disruptivos, como medidas de cuarentena o nuevas olas de casos. Por ejemplo, Prophet ha demostrado su efectividad al predecir el número de casos diarios durante la aparición de nuevas variantes del SARS-CoV-2 en países como Brasil, Rusia y Estados Unidos, logrando captar las dinámicas temporales complejas de los brotes\parencite{Baccega2024Sybil, Rodriguez2022DataCentric}.

Una de las características distintivas de Prophet es su capacidad para manejar datos incompletos o inconsistentes, un desafío común en el monitoreo de epidemias. A través de su implementación basada en regresión no lineal ajustada por segmentación, Prophet permite modelar cambios abruptos en las tendencias, lo cual es crítico en contextos de epidemias donde los patrones de transmisión pueden cambiar rápidamente debido a intervenciones externas\parencite{Nguyen2023NZSTGNN}.

Además, Prophet es particularmente valioso para generar predicciones a corto y mediano plazo, lo que lo convierte en una herramienta útil para la planificación de recursos de salud pública y la evaluación de políticas de intervención. Por ejemplo, se ha utilizado para predecir la carga de casos en diferentes regiones, ayudando a las autoridades locales a distribuir recursos como camas de hospital y vacunas\parencite{Baccega2024Sybil, Nguyen2023NZSTGNN}.

\subsection{Métodos de Gradient Boosting (e.g., XGBoost)}\label{section:gradient-boosting}

Los métodos de \textit{Gradient Boosting}, como XGBoost (eXtreme Gradient Boosting), se han convertido en herramientas ampliamente utilizadas en diversas tareas de predicción, incluido el pronóstico epidémico. Estos enfoques son particularmente efectivos para manejar datos estructurados y no lineales, características típicas de los conjuntos de datos epidemiológicos que involucran múltiples variables interrelacionadas, como movilidad poblacional, tasas de contagio y medidas de intervención\parencite{Rodriguez2022DataCentric, Nguyen2023NZSTGNN}.

XGBoost, uno de los métodos más populares dentro de esta familia, es un algoritmo de aprendizaje supervisado que optimiza árboles de decisión mediante el uso de técnicas de \textit{boosting} de gradiente. Este enfoque construye secuencialmente una serie de árboles de decisión, donde cada árbol sucesivo corrige los errores residuales del anterior, lo que permite que el modelo aprenda patrones complejos y mejore su capacidad predictiva\parencite{Nguyen2023NZSTGNN, Stergiou2022MachineLearning}. Además, XGBoost incorpora regularización para prevenir el sobreajuste y utiliza estrategias de paralelización para reducir significativamente los tiempos de entrenamiento\parencite{Stergiou2022MachineLearning}.

En el contexto del pronóstico epidémico, XGBoost ha demostrado ser particularmente útil en escenarios de predicción a corto y mediano plazo. Por ejemplo, en estudios realizados sobre la propagación del COVID-19 en Nueva Zelanda y Europa, XGBoost se utilizó para modelar la evolución de los casos confirmados, integrando datos sobre movilidad y tasas de infección históricas\parencite{Nguyen2023NZSTGNN, Baccega2024Sybil}. La capacidad del algoritmo para manejar datos heterogéneos y combinarlos con otros enfoques, como redes neuronales gráficas, lo convierte en una herramienta poderosa para capturar las dinámicas complejas de las epidemias\parencite{Rodriguez2022DataCentric}.

Otro aspecto destacado de XGBoost es su flexibilidad para incorporar diferentes tipos de datos. Por ejemplo, en estudios recientes, el modelo se ha combinado con datos socioeconómicos y geográficos, como densidad poblacional y características de movilidad regional, para predecir con precisión los picos de infecciones y las tendencias en múltiples regiones\parencite{Nguyen2023NZSTGNN, Baccega2024Sybil}. Esta capacidad de integrar fuentes de datos diversas permite a los métodos de \textit{Gradient Boosting} abordar eficazmente los desafíos de la modelización epidemiológica\parencite{Stergiou2022MachineLearning, Rodriguez2022DataCentric}.

En términos de evaluación, los experimentos han demostrado que XGBoost ofrece un desempeño competitivo frente a métodos estadísticos tradicionales y otras técnicas de aprendizaje automático. Por ejemplo, en la comparación con modelos como ARIMA y \textit{Random Forests}, XGBoost mostró una mejor capacidad para capturar patrones no lineales en datos epidemiológicos complejos, destacando como una opción robusta para la predicción de brotes epidémicos\parencite{Nguyen2023NZSTGNN, Baccega2024Sybil, Rodriguez2022DataCentric}.

\subsection{Enfoques Híbridos}\label{subsection:hybrid-approaches}

En el ámbito de la predicción epidémica, los enfoques híbridos han emergido como una solución prometedora al combinar modelos tradicionales de compartimentos, como SEIR, con técnicas de aprendizaje automático (ML) y aprendizaje profundo (DL). Estos enfoques aprovechan las fortalezas de cada técnica para abordar las limitaciones inherentes de los métodos individuales, mejorando tanto la precisión como la capacidad predictiva en escenarios complejos y dinámicos\parencite{Rodriguez2022DataCentric, Baccega2024Sybil}.

Uno de los ejemplos más representativos es \textit{AutoSEIR}, un modelo que integra un enfoque compartimental extendido con técnicas avanzadas de aprendizaje automático. En este modelo, las ecuaciones diferenciales ordinarias (EDO) que describen la dinámica del modelo SEIR son ajustadas mediante optimización basada en diferenciación automática, lo que permite estimar parámetros clave como el número reproductivo básico (\(R_0\)) y tasas de transmisión\parencite{Rizzo2020AutoSEIR}. AutoSEIR es particularmente eficaz para modelar la evolución de enfermedades como el COVID-19, incorporando datos en tiempo real para ajustar parámetros dinámicos y predecir picos de casos con una precisión notable\parencite{Baccega2024Sybil}. Este enfoque también incluye compartimentos adicionales, como \"testeados\" y \"no testeado\", que capturan aspectos críticos del subregistro y los retrasos en los datos de salud pública\parencite{Stergiou2022MachineLearning}.

Otro enfoque híbrido destacado es el marco \textit{Sybil}, que combina modelos compartimentales sensibles a variantes con técnicas de aprendizaje automático para predecir la propagación de enfermedades infecciosas. Sybil utiliza modelos SIRDS (Susceptible-Infectado-Recuperado-Fallecido-Susceptible) y se apoya en datos epidemiológicos históricos para ajustar parámetros como tasas de infección, recuperación y mortalidad. Posteriormente, estos parámetros son utilizados como datos de entrenamiento para componentes de aprendizaje automático, que proyectan la evolución futura de los parámetros clave\parencite{Baccega2024Sybil}. Este marco ha demostrado ser eficaz en la predicción de cambios bruscos en las tendencias de la pandemia, así como en la evaluación de la prevalencia futura de nuevas variantes del virus\parencite{Nguyen2023NZSTGNN, Baccega2024Sybil}.

Además, los enfoques híbridos han explorado la integración de modelos basados en aprendizaje profundo, como LSTM y GRU, con modelos compartimentales. Por ejemplo, estudios recientes han utilizado modelos híbridos que combinan redes neuronales recurrentes con modelos compartimentales para estimar parámetros estocásticos y realizar predicciones a largo plazo con mayor precisión\parencite{Nguyen2023NZSTGNN, Shah2024COVID19FU}. Estos enfoques también permiten simular escenarios de intervención, como campañas de vacunación y restricciones de movilidad, en regiones específicas\parencite{Stergiou2022MachineLearning, Baccega2024Sybil}.

El uso de enfoques híbridos ofrece varias ventajas. Por un lado, los modelos compartimentales proporcionan una estructura matemática transparente y comprensible para describir la evolución de las epidemias, lo cual es útil para explicar los resultados a partes interesadas, como responsables políticos y profesionales de la salud. Por otro lado, las técnicas de aprendizaje automático permiten optimizar los parámetros y mejorar la capacidad predictiva al aprender directamente de datos heterogéneos y ruidosos\parencite{Rodriguez2022DataCentric, Baccega2024Sybil}. Esto es especialmente relevante en contextos donde los datos de entrada son dinámicos y no lineales, como ocurre en las epidemias modernas\parencite{Rizzo2020AutoSEIR, Baccega2024Sybil, Stergiou2022MachineLearning}.

\section{Redes Neuronales de Grafos en la Predicción Epidémica}\label{section:gnns-epidemic-forecasting}

En los últimos años, las Redes Neuronales de Grafos (GNNs, por sus siglas en inglés) han emergido como herramientas avanzadas para modelar la propagación de epidemias al capturar las complejas interacciones entre factores espaciales y temporales. Las GNNs, debido a su capacidad para operar sobre datos estructurados en forma de grafos, representan una mejora significativa respecto a los modelos tradicionales y basados en aprendizaje automático, especialmente al abordar la naturaleza heterogénea y dinámica de las epidemias. Este apartado explora los avances recientes en el uso de GNNs para la predicción epidémica, destacando sus arquitecturas, aplicaciones y casos de uso relevantes.

\subsection{Estructura y Fundamentos de las Redes Neuronales de Grafos}

% Las GNNs extienden los principios de las redes neuronales tradicionales a datos representados como grafos, donde los nodos pueden representar regiones geográficas, individuos u otras unidades, mientras que los bordes denotan interacciones, como movilidad humana o transmisión de enfermedades. Una característica clave de las GNNs es el mecanismo de \textit{message passing}, en el cual cada nodo actualiza su representación al agregar información de sus vecinos a través de iteraciones. Estas arquitecturas han evolucionado para incluir variantes como las Graph Convolutional Networks (GCN)\parencite{kipf2016semiGCN}, las Graph Attention Networks (GAT)\parencite{velivckovic2017GAT}, y las Redes Neuronales de Grafos Espacio-Temporales (STGNN)\parencite{Croft2023DutchSARS}, cada una adaptada a diferentes tipos de relaciones en datos epidemiológicos.

Las GNNs extienden los principios de las redes neuronales tradicionales a datos representados como grafos, donde los nodos pueden representar regiones geográficas, individuos u otras unidades, mientras que los bordes denotan interacciones, como movilidad humana o transmisión de enfermedades. Una característica clave de las GNNs es el mecanismo de \textit{message passing}, en el cual cada nodo actualiza su representación al agregar información de sus vecinos a través de iteraciones. Estas arquitecturas han evolucionado para incluir variantes como las Graph Convolutional Networks (GCN), las Graph Attention Networks (GAT), y las Redes Neuronales de Grafos Espacio-Temporales (STGNN), cada una adaptada a diferentes tipos de relaciones en datos epidemiológicos.

\subsection{Aplicaciones de las GNNs en la Predicción Epidémica}

Las GNNs han sido ampliamente utilizadas para modelar la propagación de enfermedades infecciosas debido a su capacidad para integrar datos espacio-temporales. En particular, se han aplicado en estudios sobre el COVID-19, el dengue y otras enfermedades transmisibles. Un enfoque común consiste en construir grafos donde los nodos representan regiones geográficas y los bordes capturan la movilidad o las interacciones entre estas regiones. Por ejemplo, el trabajo de Panagopoulos et al.\parencite{Panagopoulos_Nikolentzos_Vazirgiannis_2021} utiliza GNNs para predecir el número de casos futuros de COVID-19 en función de los patrones de movilidad y los datos históricos de casos.

\subsection{Arquitecturas Espacio-Temporales}

Las Redes Neuronales de Grafos Espacio-Temporales (STGNN) son una extensión clave que incorpora tanto relaciones espaciales como dependencias temporales en sus predicciones. Estas redes combinan módulos para capturar interacciones locales y globales en diferentes resoluciones, como lo demuestran los modelos de redes multirresolución propuestos en Nguyen et al.\parencite{Nguyen2023NZSTGNN} y Hy et al.\parencite{hy2022temporal}. Por ejemplo, el modelo ATMGNN propuesto por Nguyen et al.\parencite{Nguyen2023NZSTGNN} utiliza atención multirresolución para adaptarse a diferentes escalas espaciales y temporales, mejorando la precisión de las predicciones.

Otro avance importante es el desarrollo de modelos híbridos que combinan GNNs con modelos epidemiológicos tradicionales. Por ejemplo, HeatGNN\parencite{Zheng2024HeatGNN} integra las ecuaciones del modelo SIR en un marco de GNN, lo que permite capturar las dinámicas de transmisión heterogéneas entre ubicaciones al tiempo que mejora la generalización del modelo.

\subsection{Casos de Estudio y Resultados Relevantes}

Varios estudios han demostrado la efectividad de las GNNs en contextos reales. El trabajo de Panagopoulos et al.\parencite{Panagopoulos_Nikolentzos_Vazirgiannis_2021} emplea GNNs para modelar la propagación de COVID-19 en países europeos, utilizando un enfoque de \textit{transfer learning} para mejorar las predicciones en países con datos limitados. De manera similar, Croft et al.\parencite{Croft2023DutchSARS} utiliza GNNs espacio-temporales para predecir infecciones semanales en los Países Bajos, incorporando datos de movilidad y características demográficas.

Además, los modelos recientes han explorado la incorporación de datos socioeconómicos y de movilidad, como se detalla en Hy et al.\parencite{hy2022temporal}, demostrando que la integración de información geoespacial mejora significativamente el rendimiento predictivo.

\subsection{Ventajas y Contribuciones de las GNNs}

Las GNNs ofrecen varias ventajas sobre los enfoques tradicionales y basados en aprendizaje profundo. En primer lugar, su capacidad para modelar explícitamente las interacciones espaciales y temporales permite capturar las dinámicas subyacentes de transmisión epidémica de manera más realista\parencite{Croft2023DutchSARS}. En segundo lugar, las GNNs son flexibles y pueden integrarse con datos heterogéneos, como movilidad, características demográficas y políticas públicas, lo que las hace ideales para contextos complejos como el cubano. Finalmente, su capacidad para operar a diferentes escalas, desde nodos individuales hasta macroregiones, las posiciona como una herramienta clave para la planificación y respuesta ante epidemias\parencite{Zheng2024HeatGNN, Nguyen2023NZSTGNN}.

En resumen, las GNNs representan un avance significativo en la predicción epidémica al superar las limitaciones de los modelos tradicionales y basados en aprendizaje profundo, ofreciendo una solución robusta y escalable para modelar epidemias en entornos dinámicos y heterogéneos.