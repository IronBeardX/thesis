\documentclass[12pt,oneside]{uhthesis}
\usepackage{subfigure}
\usepackage[ruled,lined,linesnumbered,titlenumbered,algochapter,spanish,onelanguage]{algorithm2e}
\usepackage{amsmath}
\usepackage{amssymb}
\usepackage{amsbsy}
\usepackage{caption,booktabs}
\captionsetup{ justification = centering }
%\usepackage{mathpazo}
\usepackage{float}
\setlength{\marginparwidth}{2cm}
\usepackage{todonotes}
\usepackage{listings}
\usepackage{xcolor}
\usepackage{multicol}
\usepackage{graphicx}
\floatstyle{plaintop}
\restylefloat{table}
\addbibresource{Bibliography.bib}
% \setlength{\parskip}{\baselineskip}%
\renewcommand{\tablename}{Tabla}
\renewcommand{\listalgorithmcfname}{Índice de Algoritmos}
%\dontprintsemicolon

\SetAlgoNoEnd\definecolor{codegreen}{rgb}{0,0.6,0}
\definecolor{codegray}{rgb}{0.5,0.5,0.5}
\definecolor{codepurple}{rgb}{0.58,0,0.82}
\definecolor{backcolour}{rgb}{0.95,0.95,0.92}

\lstdefinestyle{mystyle}{
    backgroundcolor=\color{backcolour},   
    commentstyle=\color{codegreen},
    keywordstyle=\color{purple},
    numberstyle=\tiny\color{codegray},
    stringstyle=\color{codepurple},
    basicstyle=\ttfamily\footnotesize,
    breakatwhitespace=false,         
    breaklines=true,                 
    captionpos=b,                    
    keepspaces=true,                 
    numbers=left,                    
    numbersep=5pt,                  
    showspaces=false,                
    showstringspaces=false,
    showtabs=false,                  
    tabsize=4
}

\lstset{style=mystyle}

\title{Título de la tesis}
\author{\\\vspace{0.25cm}Nombre del autor}
\advisor{\\\vspace{0.25cm}Nombre del primer tutor\\\vspace{0.2cm}Nombre del segundo tutor}
\degree{Licenciado en (Matemática o Ciencia de la Computación)}
\faculty{Facultad de Matemática y Computación}
\date{Fecha\\\vspace{0.25cm}\href{https://github.com/username/repo}{github.com/username/repo}}
\logo{Graphics/uhlogo}

\makenomenclature\renewcommand{\vec}[1]{\boldsymbol{#1}}
\newcommand{\diff}[1]{\ensuremath{\mathrm{d}#1}}
\newcommand{\me}[1]{\mathrm{e}^{#1}}
\newcommand{\pf}{\mathfrak{p}}
\newcommand{\qf}{\mathfrak{q}}
%\newcommand{\kf}{\mathfrak{k}}
\newcommand{\kt}{\mathtt{k}}
\newcommand{\mf}{\mathfrak{m}}
\newcommand{\hf}{\mathfrak{h}}
\newcommand{\fac}{\mathrm{fac}}
\newcommand{\maxx}[1]{\max\left\{ #1 \right\} }
\newcommand{\minn}[1]{\min\left\{ #1 \right\} }
\newcommand{\lldpcf}{1.25}
\newcommand{\nnorm}[1]{\left\lvert#1 \right\rvert}
\renewcommand{\lstlistingname}{Ejemplo de código}
\renewcommand{\lstlistlistingname}{Ejemplos de código}

\begin{document}

\frontmatter
\maketitle

\include{FrontMatter/Dedication}
\include{FrontMatter/Thanks}
\include{FrontMatter/SupervisorOpinion}
\include{FrontMatter/Abstract}
\include{FrontMatter/Contents}

\mainmatter\chapter*{Introducción}\label{chapter:introduction}
\addcontentsline{toc}{chapter}{Introducción}

La pandemia de COVID-19 ha resaltado la importancia de contar con herramientas efectivas para la predicción y control de brotes epidémicos. Desde su aparición a finales de 2019, el virus SARS-CoV-2 ha provocado impactos sociales y económicos globales sin precedentes. En este contexto, los avances en inteligencia artificial y aprendizaje profundo han ofrecido nuevas oportunidades para abordar problemas complejos en la modelación epidemiológica. Este trabajo se centra en la aplicación de \textit{Graph Neural Networks} (GNNs) y \textit{Transfer Learning} para la predicción epidémica, con un enfoque particular en el caso de Cuba.

\section*{Epidemias: Visión Histórica y Conceptual}

Las epidemias han sido una constante amenaza para la humanidad a lo largo de la historia, afectando profundamente la salud pública, las economías y las estructuras sociales. Se definen como la aparición y propagación de una enfermedad en una población específica en un período de tiempo determinado, con una frecuencia superior a la esperada en condiciones normales. A lo largo de los siglos, diferentes epidemias han dejado una marca indeleble en las sociedades, desde la \textit{Peste Negra} en Europa hasta la \textit{Gripe Española} de 1918 y más recientemente, la pandemia de \textit{COVID-19}. Cada una de estas crisis ha evidenciado la necesidad de desarrollar métodos más efectivos para la predicción y control de brotes epidémicos, dado que la propagación de enfermedades infecciosas está influenciada por una compleja interacción de factores sociales, geográficos y temporales.

En el caso de Cuba, el sistema de salud pública ha jugado un papel crucial en la mitigación de brotes epidémicos. Sin embargo, a pesar de los avances en medicina y tecnología, los desafíos derivados de la limitada disponibilidad de recursos y las barreras económicas han condicionado la efectividad de las estrategias de control. El país ha enfrentado brotes significativos de enfermedades infecciosas, entre los que destacan el \textit{dengue}, la \textit{dengue hemorrágico}, \textit{HIV} y \textit{H1N1}, los cuales se han convertido en hitos de la historia epidemiológica cubana. En cada uno de estos casos, las respuestas de salud pública han sido decisivas para minimizar el impacto social y sanitario. No obstante, la variabilidad en la incidencia de estas enfermedades, junto con los cambios en los patrones de movilidad de la población y la influencia de factores socioeconómicos, ha resaltado la necesidad de modelos predictivos más sofisticados.

La dinámica de propagación de las enfermedades infecciosas es inherentemente compleja. Factores como los períodos de incubación, los cuales varían significativamente entre diferentes enfermedades, y las interacciones sociales dentro de una población, son determinantes claves en la evolución de un brote epidémico. Los períodos de gestación, que definen el tiempo que transcurre desde la exposición al patógeno hasta la manifestación de los primeros síntomas, son fundamentales para la predicción de la propagación de una epidemia. Enfermedades como la \textit{COVID-19} tienen una fase de incubación de entre 2 y 14 días, mientras que otras, como el \textit{dengue}, pueden presentar períodos de incubación de 3 a 14 días. Esta variabilidad en los períodos de gestación plantea desafíos significativos para la modelación epidemiológica, ya que deben tenerse en cuenta tanto los factores temporales como los sociales en la propagación de la enfermedad.

El \textit{dengue}, particularmente el \textit{dengue hemorrágico}, ha sido una de las enfermedades más impactantes en Cuba, especialmente en las décadas de los 80 y 90, cuando el país enfrentó varios brotes graves. La capacidad de respuesta del sistema de salud cubano, que incluyó medidas de control como la fumigación masiva y la educación comunitaria, logró contener los brotes, pero los cambios en las condiciones ambientales y los patrones de movilidad han incrementado la complejidad de la transmisión en años posteriores. Investigaciones como las de \textcite{dengue-cuba-2010} y \textcite{hemorrhagic-dengue-2015} han demostrado que la modelación precisa de la propagación del dengue debe tener en cuenta factores de movilidad y condiciones socioeconómicas, elementos que a menudo son difíciles de integrar en modelos tradicionales.

La \textit{HIV} en Cuba, aunque menos prevalente que en otras regiones, ha tenido un impacto significativo en la salud pública debido a la alta tasa de diagnóstico y tratamiento. Los brotes de \textit{HIV} en los años 80 y 90 provocaron un aumento en los esfuerzos de prevención y educación sanitaria, con un enfoque en el diagnóstico temprano y el tratamiento. En este contexto, la modelación de la propagación de enfermedades como \textit{HIV} ha sido clave para la implementación de políticas de prevención efectivas, y los modelos predictivos han jugado un papel crucial en la planificación de recursos y en la asignación de esfuerzos preventivos.

En cuanto al \textit{H1N1}, la epidemia que se presentó en 2009 también dejó importantes lecciones sobre la respuesta de salud pública en Cuba. Aunque el país implementó medidas rápidas para contener el virus, como el aislamiento de casos sospechosos y la distribución de antivirales, la propagación de la enfermedad se vio influenciada por la rápida movilidad de las personas, un factor exacerbado por el turismo y la globalización. La capacidad para predecir y contener brotes de \textit{H1N1} se vio limitada por la falta de modelos de predicción dinámicos que integraran los patrones de movilidad y las características socioeconómicas de las diferentes regiones del país.

La creciente complejidad de la propagación de enfermedades infecciosas, junto con la limitada disponibilidad de datos precisos y actualizados, subraya la necesidad de utilizar enfoques más sofisticados en la predicción epidémica. El análisis de factores espaciales, temporales y sociales, como los realizados en estudios sobre el \textit{COVID-19} en Cuba \textcite{covid-cuba-2020}, resalta la importancia de incorporar herramientas avanzadas como las \textit{Graph Neural Networks} (GNNs), que pueden integrar redes espaciales y temporales, así como datos heterogéneos de diversas fuentes, para modelar con mayor precisión la propagación de enfermedades en contextos complejos.

El análisis de estas epidemias históricas y su impacto en Cuba establece un contexto crucial para comprender los desafíos y la necesidad de enfoques más avanzados en la predicción de epidemias. Las experiencias pasadas proporcionan un marco de referencia para la identificación de patrones recurrentes y la comprensión de la interacción entre factores geográficos, sociales y económicos, que son esenciales para la predicción eficaz de la propagación de enfermedades infecciosas en el futuro.

\section*{Antecedentes del Problema Científico}
La modelación epidemiológica tradicional, basada en la asumción de que un sistema complejo puede comprenderse examinando el funcionamiento de sus partes y la manera en que se juntan, como los modelos SEIR, tiene limitaciones significativas cuando se aplican a regiones con alta heterogeneidad espacial y temporal en los patrones de transmisión. La integración de datos espaciales y temporales mediante redes neuronales profundas, particularmente GNNs, ha demostrado ser una herramienta poderosa para abordar estas limitaciones. Además, la técnica de \textit{Transfer Learning} permite aprovechar conocimientos adquiridos en dominios relacionados para mejorar el desempeño en escenarios con datos limitados.

En el contexto cubano, la historia epidemiológica es rica y compleja. Desde la introducción de enfermedades infecciosas durante la colonización hasta las campañas exitosas de vacunación contra la poliomielitis en el siglo XX, Cuba ha enfrentado numerosos retos sanitarios. Las epidemias de dengue, fiebre amarilla y poliomielitis han marcado hitos en la salud pública del país. La pandemia de COVID-19 destacó la combinación de un sistema de salud con recursos limitados y el compromiso estatal con la salud pública, generando un modelo de respuesta basado en aislamiento estricto, movilización comunitaria y el desarrollo de vacunas nacionales como Abdala y Soberana. Estas estrategias, aunque efectivas en muchos aspectos, subrayan la necesidad de herramientas modernas que puedan abordar la complejidad y la dinámica cambiante de las epidemias.

Además, la predicción precisa de la propagación del COVID-19 en Cuba enfrenta retos significativos. Entre ellos, destaca la escasez de información detallada y consistente sobre movilidad poblacional, redes de interacción social y registros sanitarios, lo cual dificulta el entrenamiento de modelos robustos y generalizables. También se observa una heterogeneidad regional marcada, que incluye diferencias en densidad poblacional, comportamiento social, acceso a recursos sanitarios y niveles socioeconómicos. Esta complejidad requiere modelos que puedan capturar y representar estas variaciones de manera precisa.

\section*{Presentación del Problema}
El reto central de esta investigación radica en desarrollar un modelo capaz de predecir la dinámica de la transmisión de enfermedades infecciosas en Cuba, aprovechando datos disponibles como movimientos poblacionales y condiciones socioeconómicas. Este enfoque combina las capacidades de representación de las GNNs con la flexibilidad del \textit{Transfer Learning} para ajustar el modelo a las particularidades del contexto cubano.

La predicción precisa de la propagación del COVID-19 en Cuba enfrenta retos significativos. Entre ellos, destaca la escasez de información detallada y consistente sobre movilidad poblacional, redes de interacción social y registros sanitarios, lo cual dificulta el entrenamiento de modelos robustos y generalizables. Además, existen notables diferencias sociales y demográficas entre regiones, incluyendo densidad poblacional, comportamiento social, acceso a recursos sanitarios y niveles socioeconómicos. Estos factores heterogéneos influyen de manera desigual en la dinámica de transmisión del virus y requieren ser modelados cuidadosamente para evitar generalizaciones erróneas.

\section*{Relevancia, Novedad e Importancia Práctica-Teórica}
Este trabajo aporta a la literatura existente al aplicar GNNs en un contexto geográfico subexplorado, combinando enfoques novedosos como el uso de \textit{GNN's espacio-temporales} y transferencia de conocimiento. Las aplicaciones prácticas incluyen mejoras en la planificación de recursos de salud y la capacidad de respuesta ante emergencias sanitarias.

Además, esta investigación tiene el potencial de desarrollar herramientas predictivas adaptables que mejoren la gestión de futuras epidemias. La integración de GNNs con transferencia de aprendizaje representa una contribución novedosa al campo, al combinar enfoques teóricos avanzados con aplicaciones prácticas. Este trabajo también podría tener implicaciones más amplias en el análisis de redes complejas y en la formulación de políticas públicas para fortalecer la resiliencia sanitaria. Además, se busca contribuir al establecimiento de un marco metodológico replicable que pueda ser utilizado en otras naciones insulares con características similares.

\section*{Objetivos de la Investigación}
\subsection*{Pregunta científica}
¿ Es posible disponer de un modelo basado en \textit{Graph Neural Networks} y textit{Transfer Learning} para la predicción de brotes epidémicos, integrado con datos espacio-temporales?

\subsection*{Estructura de la tesis}
Este trabajo se organiza en seis capítulos, incluyendo la introducción y la conclusión. El Capítulo 2 presenta el marco teórico y conceptual, abordando los fundamentos de la modelación epidemiológica y las técnicas de aprendizaje profundo aplicadas a la predicción de brotes epidémicos. El Capítulo 3 revisa la literatura existente sobre GNNs y Transfer Learning en el contexto de la epidemiología, destacando los avances recientes y las aplicaciones prácticas. El Capítulo 4 describe la metodología propuesta, incluyendo la selección de datos, la arquitectura del modelo y las métricas de evaluación. El Capítulo 5 presenta los resultados de la implementación del modelo en un conjunto de datos de prueba y discute las implicaciones prácticas de los hallazgos. Finalmente, el Capítulo 6 resume las conclusiones y recomendaciones derivadas de la investigación, así como las posibles áreas de desarrollo futuro.

\subsection*{Objetivo General}
Diseñar e implementar un modelo basado en \textit{Graph Neural Networks} y \textit{Transfer Learning} para la predicción de brotes epidémicos en Cuba, integrando datos espaciales y temporales.

\subsection*{Objetivos Específicos}
\begin{itemize}
    \item Analizar las técnicas actuales de modelación epidémica basadas en GNNs y su aplicabilidad al contexto cubano.
    \item Implementar una arquitectura \textit{espacio-temporal} adaptada a datos locales, considerando movilidad y características socioeconómicas.
    \item Evaluar el impacto del \textit{Transfer Learning} en la mejora de la predicción cuando se usan datos limitados.
    \item Validar el modelo propuesto utilizando datos históricos de epidemias en Cuba.
    \item Identificar potenciales áreas de aplicación del modelo para mejorar la preparación ante futuras emergencias sanitarias.
\end{itemize}
\chapter{Estado del Arte}\label{chapter:state-of-the-art}

La revisión de la literatura científica es un componente fundamental en cualquier investigación académica, ya que permite situar el trabajo en el contexto del conocimiento existente, identificar avances recientes y reconocer las limitaciones o brechas que aún persisten. En este capítulo, se realiza un análisis detallado del estado del arte relacionado con la predicción epidémica, haciendo énfasis en los métodos tradicionales y modernos, como el uso de \textit{Graph Neural Networks} (GNNs) y \textit{Transfer Learning}. 

El objetivo principal de este capítulo es proporcionar una base conceptual y técnica que sustente el desarrollo del modelo propuesto en esta tesis. Para ello, se abordan cuatro áreas principales: una introducción histórica y conceptual sobre epidemias, los enfoques tradicionales y basados en aprendizaje automático para la predicción de brotes, las aplicaciones específicas de GNNs en este ámbito, y el rol del \textit{Transfer Learning} como técnica para superar limitaciones de datos en escenarios específicos. Adicionalmente, se incluye un análisis crítico de estudios recientes relevantes para este trabajo y se destacan las oportunidades de investigación que guían el desarrollo de esta tesis.

Este capítulo está estructurado de la siguiente manera: en la Sección 2.1, se presenta un panorama general sobre las epidemias, sus impactos y las complejidades inherentes a su modelación. En la Sección 2.2, se revisan los métodos tradicionales de predicción epidémica, sus aplicaciones y limitaciones. La Sección 2.3 introduce las técnicas basadas en aprendizaje automático, con un énfasis particular en las redes neuronales profundas. En la Sección 2.4, se exploran las arquitecturas y aplicaciones de GNNs en modelación epidemiológica, mientras que en la Sección 2.5 se analiza la relevancia del \textit{Transfer Learning} y su implementación en GNNs. Finalmente, en la Sección 2.6, se discuten los principales vacíos en la literatura actual y las oportunidades de investigación que aborda este trabajo.

Este análisis no solo busca sintetizar el conocimiento disponible, sino también resaltar las áreas donde se requiere mayor investigación, particularmente en contextos con recursos limitados y características socioeconómicas particulares, como es el caso de Cuba.

\section{Epidemias: Visión Histórica y Conceptual}\label{section:epidemics-history-conceptual}

Las epidemias han sido una constante amenaza para la humanidad a lo largo de la historia, afectando profundamente la salud pública, las economías y las estructuras sociales. Se definen como la aparición y propagación de una enfermedad en una población específica en un período de tiempo determinado, con una frecuencia superior a la esperada en condiciones normales. A lo largo de los siglos, diferentes epidemias han dejado una marca indeleble en las sociedades, desde la \textit{Peste Negra} en Europa hasta la \textit{Gripe Española} de 1918 y más recientemente, la pandemia de \textit{COVID-19}. Cada una de estas crisis ha evidenciado la necesidad de desarrollar métodos más efectivos para la predicción y control de brotes epidémicos, dado que la propagación de enfermedades infecciosas está influenciada por una compleja interacción de factores sociales, geográficos y temporales.

En el caso de Cuba, el sistema de salud pública ha jugado un papel crucial en la mitigación de brotes epidémicos. Sin embargo, a pesar de los avances en medicina y tecnología, los desafíos derivados de la limitada disponibilidad de recursos y las barreras económicas han condicionado la efectividad de las estrategias de control. El país ha enfrentado brotes significativos de enfermedades infecciosas, entre los que destacan el \textit{dengue}, la \textit{dengue hemorrágico}, \textit{HIV} y \textit{H1N1}, los cuales se han convertido en hitos de la historia epidemiológica cubana. En cada uno de estos casos, las respuestas de salud pública han sido decisivas para minimizar el impacto social y sanitario. No obstante, la variabilidad en la incidencia de estas enfermedades, junto con los cambios en los patrones de movilidad de la población y la influencia de factores socioeconómicos, ha resaltado la necesidad de modelos predictivos más sofisticados.

La dinámica de propagación de las enfermedades infecciosas es inherentemente compleja. Factores como los períodos de incubación, los cuales varían significativamente entre diferentes enfermedades, y las interacciones sociales dentro de una población, son determinantes claves en la evolución de un brote epidémico. Los períodos de gestación, que definen el tiempo que transcurre desde la exposición al patógeno hasta la manifestación de los primeros síntomas, son fundamentales para la predicción de la propagación de una epidemia. Enfermedades como la \textit{COVID-19} tienen una fase de incubación de entre 2 y 14 días, mientras que otras, como el \textit{dengue}, pueden presentar períodos de incubación de 3 a 14 días. Esta variabilidad en los períodos de gestación plantea desafíos significativos para la modelación epidemiológica, ya que deben tenerse en cuenta tanto los factores temporales como los sociales en la propagación de la enfermedad.

El \textit{dengue}, particularmente el \textit{dengue hemorrágico}, ha sido una de las enfermedades más impactantes en Cuba, especialmente en las décadas de los 80 y 90, cuando el país enfrentó varios brotes graves. La capacidad de respuesta del sistema de salud cubano, que incluyó medidas de control como la fumigación masiva y la educación comunitaria, logró contener los brotes, pero los cambios en las condiciones ambientales y los patrones de movilidad han incrementado la complejidad de la transmisión en años posteriores. Investigaciones como las de \textcite{dengue-cuba-2010} y \textcite{hemorrhagic-dengue-2015} han demostrado que la modelación precisa de la propagación del dengue debe tener en cuenta factores de movilidad y condiciones socioeconómicas, elementos que a menudo son difíciles de integrar en modelos tradicionales.

La \textit{HIV} en Cuba, aunque menos prevalente que en otras regiones, ha tenido un impacto significativo en la salud pública debido a la alta tasa de diagnóstico y tratamiento. Los brotes de \textit{HIV} en los años 80 y 90 provocaron un aumento en los esfuerzos de prevención y educación sanitaria, con un enfoque en el diagnóstico temprano y el tratamiento. En este contexto, la modelación de la propagación de enfermedades como \textit{HIV} ha sido clave para la implementación de políticas de prevención efectivas, y los modelos predictivos han jugado un papel crucial en la planificación de recursos y en la asignación de esfuerzos preventivos.

En cuanto al \textit{H1N1}, la epidemia que se presentó en 2009 también dejó importantes lecciones sobre la respuesta de salud pública en Cuba. Aunque el país implementó medidas rápidas para contener el virus, como el aislamiento de casos sospechosos y la distribución de antivirales, la propagación de la enfermedad se vio influenciada por la rápida movilidad de las personas, un factor exacerbado por el turismo y la globalización. La capacidad para predecir y contener brotes de \textit{H1N1} se vio limitada por la falta de modelos de predicción dinámicos que integraran los patrones de movilidad y las características socioeconómicas de las diferentes regiones del país.

La creciente complejidad de la propagación de enfermedades infecciosas, junto con la limitada disponibilidad de datos precisos y actualizados, subraya la necesidad de utilizar enfoques más sofisticados en la predicción epidémica. El análisis de factores espaciales, temporales y sociales, como los realizados en estudios sobre el \textit{COVID-19} en Cuba \textcite{covid-cuba-2020}, resalta la importancia de incorporar herramientas avanzadas como las \textit{Graph Neural Networks} (GNNs), que pueden integrar redes espaciales y temporales, así como datos heterogéneos de diversas fuentes, para modelar con mayor precisión la propagación de enfermedades en contextos complejos.

El análisis de estas epidemias históricas y su impacto en Cuba establece un contexto crucial para comprender los desafíos y la necesidad de enfoques más avanzados en la predicción de epidemias. Las experiencias pasadas proporcionan un marco de referencia para la identificación de patrones recurrentes y la comprensión de la interacción entre factores geográficos, sociales y económicos, que son esenciales para la predicción eficaz de la propagación de enfermedades infecciosas en el futuro.

\noindent
Estas lecciones del pasado, combinadas con el desarrollo de nuevas tecnologías y métodos, como el uso de GNNs y \textit{Transfer Learning}, ofrecen una oportunidad para mejorar la precisión y la adaptabilidad de los modelos predictivos, lo que puede ser crucial para la gestión de futuras crisis sanitarias, especialmente en contextos con recursos limitados como el cubano.

\section{Métodos Tradicionales de Predicción Epidémica}\label{section:traditional-methods}

La predicción de epidemias ha sido una prioridad en la investigación epidemiológica, especialmente durante los brotes de enfermedades infecciosas, como el dengue, la fiebre amarilla, y más recientemente el COVID-19. Tradicionalmente, los modelos de predicción epidémica se han basado en métodos matemáticos y estadísticos que intentan capturar la dinámica de propagación de enfermedades a través de diferentes enfoques. En este sentido, se han propuesto varios tipos de modelos para la predicción y el control de brotes epidémicos, destacándose los modelos de compartimentos, los modelos basados en agentes y los enfoques estadísticos. 

En este apartado, se proporciona una visión general de los métodos tradicionales más utilizados para la predicción epidémica. A continuación, se presenta una clasificación gráfica de estos métodos, seguida de un análisis detallado de cada uno de ellos, sus ventajas y limitaciones, con el fin de establecer un contexto para los métodos basados en \textit{Graph Neural Networks} (GNNs) y \textit{Transfer Learning}, que serán discutidos más adelante.

\begin{figure}[H]
\centering
%TODO: \includegraphics[width=0.8\textwidth]{traditional_methods_classification.png}
\caption{Clasificación de los métodos tradicionales de predicción epidémica.}
\end{figure}

Los métodos tradicionales se dividen principalmente en tres categorías: (i) \textit{Modelos de compartimentos}, (ii) \textit{Modelos basados en agentes}, y (iii) \textit{Enfoques estadísticos}. Cada uno de estos enfoques tiene aplicaciones específicas dependiendo del contexto de la epidemia y la disponibilidad de datos.

\subsection{Modelos de Compartimentos: El Modelo SEIR y sus Variantes}\label{section:seir-model}

El modelo \textit{SEIR} (Susceptibles, Expuestos, Infectados y Recuperados) es uno de los enfoques más antiguos y utilizados en la epidemiología matemática para modelar la propagación de enfermedades infecciosas. Introducido por Kermack y McKendrick en 1927, este modelo divide a la población en cuatro compartimentos básicos: Susceptibles, Expuestos, Infectados y Recuperados. Cada compartimento representa una etapa en la que un individuo se encuentra en relación con la enfermedad, y las personas pasan de un estado a otro conforme interactúan con individuos infectados. Este modelo es particularmente útil para describir enfermedades con un período de incubación, en las cuales un individuo infectado pasa por una fase \textit{expuesta} antes de volverse infeccioso. 

La estructura básica del modelo SEIR se describe mediante un sistema de ecuaciones diferenciales ordinarias (EDOs), donde la población total se distribuye entre los cuatro compartimentos mencionados. La dinámica de estas transiciones se expresa de la siguiente forma:

\[
\frac{dS}{dt} = -\beta \frac{S I}{N}
\]
\[
\frac{dE}{dt} = \beta \frac{S I}{N} - \sigma E
\]
\[
\frac{dI}{dt} = \sigma E - \gamma I
\]
\[
\frac{dR}{dt} = \gamma I
\]

En este sistema, \(S\) representa a la población susceptible a la enfermedad, \(E\) a los individuos expuestos al patógeno pero aún no infecciosos, \(I\) a los individuos infectados e infecciosos, y \(R\) a los individuos que se han recuperado o han muerto. Los parámetros \(\beta\), \(\sigma\) y \(\gamma\) representan las tasas de transmisión, la tasa de transición del compartimento expuesto al infectado, y la tasa de recuperación de los infectados, respectivamente.

El modelo SEIR es particularmente adecuado para enfermedades con un período de incubación conocido, como el \textit{COVID-19}, en el que se sabe que el período de incubación varía entre 2 y 14 días. Este modelo ha sido ampliamente utilizado en la predicción de la propagación de diversas enfermedades infecciosas debido a su simplicidad y a la facilidad con que se puede ajustar a diferentes contextos epidemiológicos.

A pesar de su simplicidad, el modelo SEIR se basa en varias suposiciones que limitan su aplicabilidad en contextos más complejos. En primer lugar, asume que toda la población es homogénea, lo que significa que todos los individuos tienen la misma probabilidad de ser infectados, sin considerar las diferencias en edad, comportamiento social o condiciones socioeconómicas. Además, el modelo SEIR supone que los contactos entre individuos son aleatorios y homogéneos, lo cual no refleja la estructura de redes sociales complejas en la que las personas interactúan de manera no uniforme. Tampoco toma en cuenta las características espaciales y geográficas de la población, que son fundamentales para modelar la propagación de enfermedades en contextos como el de Cuba, donde existen diferencias significativas en densidad poblacional y acceso a servicios de salud.

A lo largo del tiempo, han surgido diversas variantes del modelo SEIR con el objetivo de superar algunas de estas limitaciones. Una de las principales modificaciones es el modelo SEIRS (Susceptible, Expuesto, Infectado, Recuperado, Susceptible), que introduce la posibilidad de que los individuos recuperados puedan volver a ser susceptibles después de un cierto período de tiempo, lo que es característico de enfermedades en las que la inmunidad es temporal, como ocurre con la gripe estacional. Esta variante permite modelar con mayor precisión enfermedades cuya inmunidad no dura toda la vida, lo cual es relevante en el caso de enfermedades como la \textit{influenza}.

Otro modelo que ha ganado relevancia en las últimas décadas es el SEIR con vacunación. En esta variante, se incorpora un compartimento adicional para individuos vacunados, lo que permite evaluar el impacto de las campañas de vacunación en la propagación de enfermedades. El modelo SEIR con vacunación se utiliza para predecir cómo las tasas de inmunización pueden modificar la dinámica de la transmisión, un enfoque esencial para entender la efectividad de las políticas de vacunación masiva, como las implementadas en respuesta a la pandemia de \textit{COVID-19}. En este modelo, la tasa de vacunación se integra en las ecuaciones, permitiendo calcular la proporción de la población que se vuelve inmune debido a la vacunación.

Por otro lado, el modelo SEIR espacialmente explícito introduce un enfoque que tiene en cuenta las diferencias geográficas y de movilidad en la población. En lugar de asumir que la población es homogénea, este modelo permite representar a los individuos en diferentes localizaciones geográficas y modelar los movimientos entre estas localizaciones, lo que es esencial en contextos donde la propagación de la enfermedad depende fuertemente de la movilidad de las personas, como ocurre en los países insulares o en regiones con alta interacción entre zonas urbanas y rurales. Esta variante se ha vuelto crucial para la modelización de enfermedades en contextos globalizados y urbanizados, donde las interacciones entre diferentes áreas geográficas tienen un impacto significativo en la propagación del patógeno.

Aunque las variantes del modelo SEIR ofrecen mayor flexibilidad y permiten una representación más precisa de la propagación de las enfermedades en escenarios complejos, aún existen importantes limitaciones. La principal de ellas es que estos modelos siguen basándose en suposiciones de homogeneidad en ciertos aspectos, como la tasa de contacto o la movilidad, lo cual puede no ser realista en poblaciones diversas. Además, la falta de datos detallados sobre la estructura social y las interacciones específicas entre subgrupos dentro de la población limita la capacidad de estos modelos para predecir con precisión la propagación de enfermedades en contextos de alta heterogeneidad.

Otro desafío importante es la incapacidad de estos modelos para integrar en tiempo real intervenciones como el distanciamiento social, el cierre de fronteras o las campañas de vacunación. Aunque se han propuesto modificaciones para incorporar estas intervenciones, la efectividad de las mismas depende de una respuesta dinámica que pueda adaptarse rápidamente a los cambios en la propagación de la enfermedad.

En resumen, aunque el modelo SEIR y sus variantes han sido herramientas fundamentales en la predicción y el control de epidemias, sus limitaciones en cuanto a la capacidad para manejar la heterogeneidad espacial y temporal, la falta de datos detallados y la incapacidad de adaptarse rápidamente a nuevas intervenciones subrayan la necesidad de enfoques más avanzados. Modelos como las \textit{Graph Neural Networks} (GNNs), que pueden incorporar datos espaciales, temporales y sociales, representan una solución prometedora para superar estas limitaciones, especialmente en contextos como el cubano, donde los factores geográficos y socioeconómicos juegan un papel crucial en la propagación de enfermedades.

\subsection{Modelos Basados en Agentes}\label{section:agent-based-models}

Los \textit{Modelos Basados en Agentes} (ABMs) han emergido como una herramienta poderosa en la predicción y modelado de epidemias, especialmente cuando se desea capturar la heterogeneidad espacial y temporal de las interacciones entre individuos. A diferencia de los modelos compartimentales como el SEIR, que tratan a la población como homogénea, los ABMs modelan a los individuos como entidades autónomas con características específicas, que interactúan entre sí dentro de un entorno determinado. Estas interacciones pueden ser influenciadas por factores como el comportamiento social, la ubicación geográfica, las políticas públicas, y la movilidad, permitiendo que los ABMs simulen de manera más realista la propagación de enfermedades en contextos complejos.

\subsubsection{Estructura Básica de un Modelo Basado en Agentes}

En un ABM, cada individuo es representado como un "agente" con atributos específicos, como la edad, el estado de salud, la ubicación y el comportamiento. Los agentes interactúan entre sí de acuerdo con reglas predefinidas, las cuales pueden ser estocásticas o determinísticas. Estas interacciones afectan la propagación de la enfermedad dentro de la población. Los agentes pueden moverse de un lugar a otro, contagiarse, infectar a otros individuos, recuperarse, o incluso morir, dependiendo de las reglas que rigen el modelo.

La principal ventaja de los ABMs es que permiten capturar la heterogeneidad individual, lo que significa que los agentes pueden tener diferentes probabilidades de infección dependiendo de sus características personales o de su entorno. Además, los ABMs permiten representar de forma explícita la estructura social y geográfica de la población, lo cual es crucial para entender cómo las enfermedades se propagan en entornos con redes de contactos complejas.

En la práctica, los ABMs se implementan mediante la simulación de miles o millones de agentes que interactúan según las reglas definidas. Estas simulaciones pueden llevarse a cabo en espacios discretos o continuos, y los modelos pueden incluir elementos como redes sociales, vecindarios, y comportamientos de movilidad, lo que permite una representación detallada de la propagación epidémica.

\subsubsection{Aplicaciones de los Modelos Basados en Agentes en la Predicción Epidémica}

Los ABMs han sido utilizados en una variedad de contextos para modelar la propagación de enfermedades. Uno de los ejemplos más conocidos es el uso de ABMs para simular la propagación de la \textit{gripe} y el \textit{COVID-19}, donde los modelos incorporan factores como la movilidad de la población, las políticas de salud pública y las interacciones sociales. En un estudio realizado por **Galvani et al. (2007)**, se utilizaron ABMs para modelar la propagación de la influenza en una población ficticia, considerando factores como la densidad de población, las tasas de contacto entre diferentes grupos de edad y las intervenciones de salud pública como la vacunación.

Los ABMs han demostrado ser especialmente útiles en contextos donde los modelos compartimentales tradicionales, como el SEIR, no pueden capturar adecuadamente la complejidad de la propagación epidémica. Por ejemplo, en **modelos de propagación espacial**, los ABMs pueden simular cómo la enfermedad se extiende de una región a otra a medida que las personas se desplazan entre diferentes áreas geográficas. Estos modelos son particularmente importantes en países con grandes diferencias regionales en cuanto a infraestructura de salud y densidad poblacional, como en el caso de Cuba, donde las interacciones sociales y los movimientos entre áreas urbanas y rurales pueden tener un impacto significativo en la dinámica de propagación de enfermedades.

En el caso del **COVID-19**, los ABMs han sido utilizados para evaluar el impacto de las políticas de distanciamiento social y otras intervenciones de salud pública. Los estudios han mostrado cómo las restricciones en la movilidad de los individuos pueden reducir significativamente la propagación del virus, pero también han destacado los retos en la implementación de estas medidas, especialmente en entornos con alta densidad de población o con movimientos transitorios (por ejemplo, turistas). Un estudio destacado realizado por **Kissler et al. (2020)** utilizó ABMs para simular la propagación del COVID-19 en diferentes escenarios de distanciamiento social, ayudando a comprender cómo las intervenciones pueden mitigar la propagación de la enfermedad en distintos contextos sociales.

\subsubsection{Ventajas de los Modelos Basados en Agentes}

Una de las principales ventajas de los ABMs es su capacidad para representar la **heterogeneidad** de las poblaciones. A diferencia de los modelos de compartimentos, donde todos los individuos se tratan como iguales, los ABMs permiten que cada agente tenga características únicas que afectan su probabilidad de contagio y de interacción. Por ejemplo, los agentes pueden ser clasificados según su edad, sexo, nivel de salud, comportamientos sociales y actividades diarias, lo que permite simular de manera más realista cómo las enfermedades se propagan a través de diferentes grupos dentro de la población.

Otra ventaja importante de los ABMs es su **flexibilidad** para incorporar una amplia gama de factores que afectan la propagación de la enfermedad. Los ABMs pueden incluir reglas para representar comportamientos de movilidad, redes sociales, contactos entre individuos, políticas públicas de control (como cuarentenas o vacunación) y variabilidad en la tasa de transmisión. Esta capacidad para incorporar múltiples factores hace que los ABMs sean particularmente útiles para modelar escenarios complejos donde otros métodos, como los modelos SEIR, no son suficientemente detallados.

Además, los ABMs permiten realizar **simulaciones de políticas públicas** de manera más precisa. Por ejemplo, se pueden simular diferentes escenarios de intervención, como la implementación de cuarentenas, la introducción de medidas de distanciamiento social o la distribución de vacunas, y evaluar el impacto de estas políticas en la propagación de la enfermedad. Esta capacidad para probar diferentes estrategias antes de implementarlas en la realidad es crucial para una planificación de respuesta más efectiva durante brotes epidémicos.

\subsubsection{Limitaciones de los Modelos Basados en Agentes}

A pesar de sus ventajas, los ABMs presentan varias limitaciones que deben ser consideradas al aplicarlos en el contexto de la predicción epidémica. Una de las principales limitaciones es su **alto costo computacional**. La simulación de miles o millones de agentes que interactúan entre sí puede ser muy exigente desde el punto de vista computacional, especialmente cuando se incluyen detalles complejos como la movilidad y las redes sociales. Esto puede hacer que los ABMs sean menos viables en situaciones donde se requiere una simulación rápida o en entornos con recursos limitados.

Además, los ABMs dependen de la disponibilidad de **datos detallados** sobre la población, las interacciones sociales y las políticas de salud pública. En contextos como el de Cuba, donde los datos sobre movilidad poblacional y redes de interacción social pueden ser limitados o imprecisos, los resultados de los ABMs pueden ser menos fiables. Esto resalta la necesidad de contar con datos de alta calidad y actualizados para que los modelos sean precisos y útiles en la toma de decisiones.

Otra limitación importante de los ABMs es que, aunque pueden capturar la heterogeneidad de la población, **la calibración** de los modelos puede ser un desafío. La falta de datos precisos sobre la distribución de ciertos atributos de la población (como las tasas de contacto, el comportamiento social, o las características de movilidad) puede llevar a resultados erróneos o poco representativos. Esto requiere que los investigadores hagan suposiciones sobre estos factores, lo que introduce incertidumbre en las predicciones.

\subsubsection{Conclusión}

En resumen, los modelos basados en agentes ofrecen un enfoque potente y flexible para modelar la propagación de enfermedades, especialmente en escenarios donde la heterogeneidad espacial y temporal juega un papel importante. Su capacidad para representar interacciones complejas entre individuos y su flexibilidad para incorporar múltiples factores hace que los ABMs sean una herramienta valiosa en la predicción epidémica. Sin embargo, sus limitaciones en términos de costos computacionales, disponibilidad de datos y calibración del modelo subrayan la necesidad de enfoques más eficientes y precisos, como las redes neuronales y el aprendizaje transferido, que pueden manejar estos desafíos y mejorar la precisión de las predicciones.

\subsection{Enfoques Estadísticos}\label{section:statistical-approaches}

Los enfoques estadísticos han sido una herramienta fundamental en la predicción epidémica, especialmente cuando se dispone de datos históricos limitados o de baja calidad. Estos métodos se basan en el análisis de datos pasados para modelar la propagación de una enfermedad y realizar predicciones sobre su comportamiento futuro. Aunque estos enfoques no intentan modelar la dinámica exacta de la transmisión de la enfermedad, ofrecen una forma eficiente de realizar predicciones rápidas en escenarios donde los recursos para la modelización detallada son escasos.

Uno de los enfoques más comunes en la epidemiología estadística es el uso de **modelos de regresión**. Estos modelos tratan de establecer una relación matemática entre los casos reportados de la enfermedad y una o más variables predictoras, como las condiciones climáticas, las políticas de intervención, o las características demográficas de la población. Los modelos de regresión pueden ser tanto lineales como no lineales, dependiendo de la naturaleza de la relación entre las variables. Por ejemplo, en el caso del análisis de la propagación del COVID-19, los modelos de regresión logística o de Poisson han sido utilizados para predecir la tasa de infección en función de factores como la densidad poblacional, la movilidad, y las intervenciones de salud pública.

Los **modelos de series temporales** son otro enfoque estadístico ampliamente utilizado para predecir la evolución de enfermedades a lo largo del tiempo. Estos modelos analizan los patrones de datos históricos, buscando tendencias y estacionalidades en la propagación de la enfermedad. Los modelos más comunes en este contexto son los **modelos ARIMA** (Autoregressive Integrated Moving Average) y sus variantes, que intentan capturar las relaciones de dependencia temporal entre los valores de los datos y las perturbaciones aleatorias. Los modelos ARIMA son particularmente útiles cuando se dispone de datos consistentes y de alta calidad sobre la evolución de la epidemia, como los reportes diarios de nuevos casos de infección. Sin embargo, su capacidad predictiva está limitada cuando los datos son escasos o no lineales, como ocurre en los primeros días de un brote epidémico.

Una extensión de los modelos de series temporales es el **modelo de suavizamiento exponencial**, que también intenta modelar las series de tiempo a través de un proceso de suavizado. Este tipo de modelos es particularmente útil cuando se desea realizar predicciones de corto plazo con un énfasis en las observaciones más recientes, ya que otorgan mayor peso a los datos más cercanos al momento actual. Esta propiedad permite a los modelos de suavizamiento exponencial adaptarse rápidamente a cambios repentinos en la dinámica de la epidemia, como un incremento repentino en el número de casos.

Aunque estos enfoques estadísticos han demostrado ser útiles en la predicción de la propagación de enfermedades, también presentan varias limitaciones importantes que deben tenerse en cuenta. En primer lugar, los enfoques estadísticos dependen en gran medida de la calidad y cantidad de los datos históricos disponibles. En contextos donde los datos sobre la propagación de la enfermedad son limitados o de baja calidad, los modelos estadísticos pueden no ser suficientemente precisos. Además, los modelos estadísticos generalmente asumen que la relación entre las variables predictoras y la propagación de la enfermedad es constante en el tiempo, lo cual no siempre es cierto en la práctica, especialmente en situaciones dinámicas como las pandemias. Las tasas de transmisión pueden cambiar a lo largo del tiempo debido a factores como la implementación de nuevas políticas, el comportamiento social de la población, o la aparición de nuevas variantes del patógeno.

Otra limitación importante de los enfoques estadísticos es su incapacidad para modelar explícitamente la heterogeneidad espacial y temporal de los brotes epidémicos. A diferencia de los modelos compartimentales como el SEIR o los modelos basados en agentes, los enfoques estadísticos no incorporan factores como la movilidad de la población o las interacciones sociales, que son esenciales para comprender cómo se propaga una enfermedad en contextos geográficamente diversos. Por ejemplo, en Cuba, la distribución geográfica de la población y las diferencias en las infraestructuras de salud entre áreas urbanas y rurales podrían tener un impacto significativo en la propagación de una enfermedad, pero este tipo de información no se considera en los modelos estadísticos convencionales.

Por último, los modelos estadísticos también enfrentan desafíos en la incorporación de intervenciones dinámicas. En escenarios de epidemias en tiempo real, las políticas de salud pública, como el distanciamiento social, las cuarentenas o la vacunación, pueden alterar significativamente la trayectoria de la epidemia. Los modelos estadísticos tradicionales no están bien equipados para manejar estos cambios dinámicos de forma flexible, ya que generalmente se ajustan a datos históricos sin considerar la capacidad de respuesta en tiempo real a las intervenciones.

A pesar de estas limitaciones, los enfoques estadísticos siguen siendo una herramienta valiosa para la predicción epidémica, especialmente cuando los datos son escasos o cuando se necesitan predicciones rápidas en fases tempranas de un brote. No obstante, la creciente complejidad de las epidemias modernas, sumada a la disponibilidad de nuevos tipos de datos y técnicas computacionales, hace necesario explorar enfoques más sofisticados, como el uso de **Graph Neural Networks** (GNNs) y **Transfer Learning**, que permiten incorporar de manera más precisa y flexible los diversos factores espaciales, temporales y sociales en la modelización de la propagación de enfermedades infecciosas.

En resumen, los enfoques estadísticos han sido fundamentales para la predicción epidémica, pero sus limitaciones, particularmente en contextos dinámicos y heterogéneos, subrayan la necesidad de métodos más avanzados. Aunque son apropiados para realizar predicciones a corto plazo basadas en datos históricos, estos enfoques carecen de la capacidad para adaptarse rápidamente a nuevas situaciones o integrar variables complejas como la movilidad poblacional o las intervenciones en tiempo real. Por lo tanto, los modelos basados en redes neuronales, que pueden manejar estos factores, representan una mejora significativa para la predicción de epidemias en escenarios como el cubano.

\subsection{Limitaciones de los Métodos Tradicionales}

A pesar de su utilidad, los métodos tradicionales de predicción epidémica presentan limitaciones significativas cuando se aplican a escenarios más complejos, como en el caso de países con características geográficas y socioeconómicas únicas, como Cuba. La principal limitación común de estos modelos es su incapacidad para manejar la heterogeneidad espacial y temporal de los brotes. La falta de datos detallados sobre la movilidad poblacional, las interacciones sociales y las características socioeconómicas dificulta la capacidad de los modelos tradicionales para predecir con precisión la propagación de enfermedades en contextos locales. Además, estos enfoques no son lo suficientemente flexibles como para adaptarse rápidamente a las dinámicas cambiantes de las epidemias, lo que limita su efectividad en situaciones de emergencia.

Por estas razones, es fundamental explorar métodos más avanzados y adaptativos, como el uso de redes neuronales profundas y aprendizaje transferido, que permitan incorporar datos heterogéneos y representar de manera más precisa las dinámicas complejas de la propagación epidémica.

\section{Métodos Basados en Aprendizaje Automático}\label{section:machine-learning-methods}

El uso de técnicas de aprendizaje automático (AA) en la predicción epidémica ha ganado popularidad en los últimos años debido a su capacidad para manejar grandes volúmenes de datos, adaptarse a la dinámica cambiante de las epidemias y modelar relaciones complejas que no son fácilmente capturadas por los enfoques tradicionales. A diferencia de los métodos clásicos como los modelos de compartimentos, que requieren suposiciones previas sobre la estructura de la propagación de la enfermedad, los métodos basados en AA pueden aprender directamente de los datos, lo que los convierte en herramientas poderosas para modelar la propagación de enfermedades en contextos dinámicos y con datos incompletos o ruidosos.

Existen diversos enfoques de AA que han sido aplicados al pronóstico epidémico, entre los cuales destacan los modelos de **redes neuronales recurrentes (LSTM)**, los **árboles de decisión aleatorios (Random Forests)** y el **aprendizaje por refuerzo (Reinforcement Learning)**. Cada uno de estos métodos tiene sus fortalezas y limitaciones, y su aplicabilidad depende del tipo de datos disponibles y de la naturaleza del brote epidémico que se desea modelar.

\subsection{Redes Neuronales Recurrentes (LSTM)}

Las **Redes Neuronales Recurrentes (RNNs)**, y en particular las variantes **LSTM (Long Short-Term Memory)**, son una de las arquitecturas más populares para modelar secuencias temporales. Estas redes son especialmente útiles para predecir eventos futuros a partir de series de tiempo, como los casos diarios de una epidemia. Las LSTM pueden capturar dependencias a largo plazo en los datos, lo que las hace particularmente adecuadas para modelar la propagación de enfermedades, donde las interacciones entre eventos pasados y futuros son clave.

En el contexto de las epidemias, las LSTM pueden ser entrenadas para predecir la propagación de enfermedades en función de datos históricos de incidencia de casos, factores socioeconómicos y de movilidad. Un ejemplo destacado es el uso de LSTMs para predecir los casos de COVID-19 en diversas regiones, donde el modelo aprende las dinámicas de la propagación de la enfermedad a partir de los datos previos de incidencia y movilidad de la población. 

Las LSTM tienen la capacidad de aprender de manera no lineal las relaciones entre los distintos factores que afectan la propagación de la enfermedad, lo que las hace mucho más flexibles que los modelos compartimentales tradicionales. Sin embargo, estas redes también presentan algunos inconvenientes. Primero, el entrenamiento de redes LSTM requiere grandes cantidades de datos para evitar el sobreajuste y asegurar que el modelo generalice bien a nuevos datos. Además, aunque las LSTM son capaces de modelar dependencias temporales, no pueden manejar explícitamente relaciones espaciales, lo que limita su capacidad para modelar epidemias que se propagan de manera heterogénea en diferentes áreas geográficas. Este problema se puede abordar integrando las LSTM con otras arquitecturas, como las redes neuronales gráficas, que son capaces de capturar las relaciones espaciales.

\subsection{Árboles de Decisión Aleatorios (Random Forests)}

El método de **Árboles de Decisión Aleatorios** (Random Forests) es una técnica de aprendizaje supervisado basada en la creación de múltiples árboles de decisión, cada uno entrenado con una muestra aleatoria de los datos. Los resultados de todos los árboles se combinan para hacer una predicción más robusta. Los Random Forests son conocidos por su capacidad para manejar datos no lineales, detectar interacciones complejas entre variables y, al mismo tiempo, evitar el sobreajuste.

En el contexto de las epidemias, los Random Forests pueden ser utilizados para predecir la probabilidad de que un área experimente un brote epidémico en función de una variedad de características, tales como la densidad poblacional, las tasas de vacunación, las medidas de control implementadas, y los factores climáticos. Un estudio relevante es el de **Husnayain et al. (2019)**, donde los Random Forests fueron empleados para predecir brotes de dengue, considerando variables como la temperatura, la humedad y los movimientos poblacionales.

Los Random Forests tienen la ventaja de ser relativamente fáciles de entrenar y no requieren un preprocesamiento extenso de los datos. Además, pueden manejar bien datos faltantes y no necesitan una normalización compleja de los datos. No obstante, una de sus limitaciones es la falta de interpretabilidad de los modelos resultantes. Aunque es posible calcular la importancia de cada variable, el modelo en su conjunto puede ser difícil de entender, lo que puede ser un desafío en situaciones donde se requiere explicabilidad para la toma de decisiones en salud pública.

\subsection{Aprendizaje por Refuerzo (Reinforcement Learning)}

El **Aprendizaje por Refuerzo** (RL) es una técnica de aprendizaje automático que se basa en la interacción con un entorno para aprender a tomar decisiones a través de la retroalimentación de recompensas o penalizaciones. En el contexto de la predicción epidémica, el aprendizaje por refuerzo puede ser utilizado para optimizar las estrategias de intervención, como la implementación de cuarentenas o la distribución de recursos médicos, con el objetivo de minimizar el impacto de un brote epidémico.

En lugar de predecir simplemente la propagación de la enfermedad, los algoritmos de RL pueden aprender a maximizar el bienestar social (por ejemplo, minimizar las infecciones y muertes) a través de políticas de intervención adaptativas. Un ejemplo de su aplicación es el uso de RL para optimizar la distribución de vacunas en diferentes regiones, basándose en la dinámica epidémica local y el comportamiento de la población.

Una de las grandes ventajas del aprendizaje por refuerzo es su capacidad para aprender políticas óptimas a medida que se desarrollan los brotes, permitiendo que las intervenciones se ajusten dinámicamente a las condiciones cambiantes. Sin embargo, el RL también enfrenta algunos retos, principalmente la necesidad de simulaciones extensivas para entrenar los modelos y la dificultad para integrar datos históricos directamente en los algoritmos de refuerzo. Además, el RL requiere de una definición clara de las recompensas y penalizaciones, lo cual puede ser complejo en el contexto de epidemias, donde los resultados no siempre son directamente medibles o inmediatos.

\subsection{Limitaciones de los Métodos Basados en Aprendizaje Automático}

A pesar de las ventajas que los métodos de aprendizaje automático ofrecen en la predicción de epidemias, también presentan limitaciones importantes. Una de las principales desventajas de estos enfoques es su dependencia de **grandes volúmenes de datos** de alta calidad. Los modelos de AA, especialmente las redes neuronales profundas, requieren grandes cantidades de datos históricos para aprender patrones de propagación precisos. En contextos como el cubano, donde los datos sobre movilidad y características socioeconómicas son limitados o imprecisos, estos modelos pueden no ser tan efectivos.

Además, los modelos de aprendizaje automático tienden a ser **"cajas negras"**, lo que significa que puede ser difícil interpretar cómo el modelo llega a sus predicciones. Esto representa un desafío en un contexto de salud pública, donde la transparencia y la explicabilidad son fundamentales para la toma de decisiones informadas y la comunicación de riesgos a la población.

Por último, los métodos de AA no están exentos de **problemas de generalización**. Aunque estos modelos pueden funcionar bien en escenarios similares a los datos con los que fueron entrenados, pueden tener dificultades para generalizar a nuevos brotes o a situaciones en las que los datos subyacentes cambian significativamente. Esto es particularmente relevante en epidemias de enfermedades emergentes, donde la falta de datos históricos dificulta la predicción precisa de su propagación.

\subsection{Conclusión}

En resumen, los métodos basados en aprendizaje automático, como las redes neuronales recurrentes (LSTM), los árboles de decisión aleatorios (Random Forests) y el aprendizaje por refuerzo (Reinforcement Learning), ofrecen un enfoque prometedor para la predicción de epidemias al aprender de los datos y adaptarse a nuevas situaciones. Sin embargo, sus limitaciones, como la necesidad de grandes cantidades de datos, la falta de interpretabilidad y los problemas de generalización, destacan la necesidad de enfoques más avanzados que puedan superar estos desafíos. Las redes neuronales gráficas (GNNs) y el aprendizaje transferido (Transfer Learning) podrían complementar estos métodos al incorporar información espacial y temporal de manera más eficiente, mejorando así la precisión y aplicabilidad de las predicciones en escenarios complejos como el de Cuba.

\chapter{Propuesta}\label{chapter:proposal}

En el contexto de la pandemia de COVID-19, predecir con precisión la dinámica de transmisión de enfermedades infecciosas se ha convertido en un desafío crítico para la gestión de la salud pública. Las estrategias tradicionales de modelización, aunque útiles en sus contextos originales, presentan limitaciones significativas al enfrentarse a la complejidad y heterogeneidad de escenarios modernos, especialmente en países con datos limitados y alta variabilidad geográfica y socioeconómica. Este capítulo presenta la propuesta metodológica de esta investigación, basada en las capacidades avanzadas de las redes neuronales de grafos espacio-temporales (\textit{Spatio-Temporal Graph Neural Networks}, STGNNs), diseñadas para superar las restricciones de los enfoques convencionales y adaptarse a la complejidad del contexto cubano.

% La propuesta toma como base los avances recientes en el campo de las STGNNs, tal como se describe en trabajos clave como \textit{Attention-based Temporal Multiresolution Graph Neural Networks (ATMGNN)}, \textit{Temporal Multiresolution Graph Neural Networks (TMGNN)} y \textit{Transfer Graph Neural Networks (T-GNN)}. Estas metodologías han demostrado ser altamente efectivas para integrar datos espacio-temporales, capturar patrones de propagación epidémica y mejorar la precisión predictiva en escenarios de alta incertidumbre. Inspirados por estas contribuciones, proponemos un modelo adaptado que combina la representación jerárquica multirresolución, la integración de datos socioeconómicos y la movilidad, y un enfoque basado en aprendizaje transferido para maximizar la utilidad de datos limitados.

% El objetivo principal de esta propuesta es desarrollar un modelo robusto y escalable que permita predecir la propagación del COVID-19 en Cuba, aprovechando tanto los datos disponibles como las estructuras espaciales y temporales subyacentes. A continuación, se detallan los fundamentos teóricos y metodológicos que sustentan esta propuesta, destacando las innovaciones clave de los modelos seleccionados y las adaptaciones necesarias para su aplicación en el contexto cubano.

% \subsection{Modelo ATMGNN: Attention-Based Temporal Multiresolution Graph Neural Networks}

% El modelo ATMGNN (\textit{Attention-based Temporal Multiresolution Graph Neural Networks}), introducido en el estudio sobre Nueva Zelanda, combina información espacial y temporal para realizar predicciones robustas de la propagación del COVID-19. Este modelo utiliza un enfoque de aprendizaje multirresolución para representar la dinámica de la pandemia a múltiples escalas geográficas y temporales. Los componentes clave del modelo son los siguientes \cite{174}:

% \begin{itemize}
%     \item \textbf{Estructura Multirresolución:} Se emplea un algoritmo de agrupamiento de datos (\textit{learning to cluster}) que permite agrupar regiones geográficas en diferentes niveles de resolución, desde nodos individuales hasta supernodos que representan ciudades o países.
%     \item \textbf{Mecanismo de Atención:} Un mecanismo de atención dinámico selecciona la resolución más relevante para cada etapa de la pandemia, adaptándose a las señales locales o globales según sea necesario.
%     \item \textbf{Incorporación de Datos Temporales:} La arquitectura incluye un módulo temporal que integra información histórica para capturar las dependencias a largo plazo en la propagación del virus.
% \end{itemize}

% Este enfoque permitió al modelo superar a múltiples métodos de referencia, incluyendo ARIMA, modelos basados en LSTM, y otros enfoques de aprendizaje profundo, tanto en términos de precisión como de robustez frente a datos ruidosos o incompletos.

% \subsection{Modelo TMGNN: Temporal Multiresolution Graph Neural Networks}

% El modelo TMGNN (\textit{Temporal Multiresolution Graph Neural Networks}) extiende las capacidades de las GNNs tradicionales mediante la introducción de representaciones multiescala para capturar tanto la información local como global en gráficos dinámicos \cite{172}. Los aspectos innovadores de esta arquitectura incluyen:

% \begin{itemize}
%     \item \textbf{Construcción Jerárquica del Grafo:} Se genera una jerarquía de gráficos mediante un módulo de agrupamiento que adapta dinámicamente la granularidad de las regiones representadas en el grafo.
%     \item \textbf{Selección Basada en Atención:} Un módulo de atención permite seleccionar las resoluciones más relevantes para optimizar las predicciones a diferentes escalas geográficas.
%     \item \textbf{Integración Temporal:} Se incorpora un modelo basado en series temporales para aprender las transiciones dinámicas entre nodos del grafo.
% \end{itemize}

% Los resultados experimentales muestran que este modelo es altamente eficaz para predecir tanto los picos como las tendencias a largo plazo de epidemias como el COVID-19 y la varicela en Europa.

% \subsection{Modelo T-GNN: Transfer Graph Neural Networks}

% El modelo T-GNN (\textit{Transfer Graph Neural Networks}) introduce técnicas de aprendizaje transferido para mejorar la precisión de las predicciones en escenarios con datos limitados \cite{173}. Este enfoque se basa en los siguientes principios:

% \begin{itemize}
%     \item \textbf{Aprendizaje Transferido:} Utiliza el método MAML (\textit{Model-Agnostic Meta-Learning}) para transferir conocimiento de países con datos más completos a países con brotes incipientes.
%     \item \textbf{Incorporación de Movilidad:} Los datos de movilidad entre regiones se representan mediante un grafo, donde los nodos corresponden a regiones y los bordes representan los flujos de población.
%     \item \textbf{Codificación de Patrones de Difusión:} El modelo aprende las dinámicas subyacentes de la transmisión epidémica al combinar datos espaciales y temporales.
% \end{itemize}

% Este enfoque demostró ser especialmente útil para predecir la evolución del COVID-19 en regiones de Europa con datos limitados, destacando el potencial del aprendizaje transferido para mejorar la precisión en escenarios de brotes secundarios.

% \subsection{Comparación y Resultados Clave}

% En conjunto, los tres modelos destacan por su capacidad para abordar las limitaciones de los métodos tradicionales y otros enfoques de aprendizaje profundo. A través de la integración de datos espacio-temporales, mecanismos de atención, y estrategias de aprendizaje transferido, estos modelos ofrecen herramientas robustas y adaptativas para la predicción epidémica. Los resultados experimentales indican mejoras significativas en métricas de precisión como el MAE (Error Absoluto Medio) y el RMSE (Raíz del Error Cuadrático Medio), consolidando su relevancia en el campo de la modelización epidemiológica.

\include{MainMatter/Implementation}

\backmatter\include{BackMatter/Conclusions}
\include{BackMatter/Recomendations}
\include{BackMatter/Bibliography}

\end{document}